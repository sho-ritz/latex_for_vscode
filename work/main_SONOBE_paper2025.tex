\documentclass[12pt,eclepsfx,a4j,twoside,openright]{./tex_files/jreport2}
%\usepackage[dvips]{graphicx}
\usepackage[dvipdfmx]{graphicx}
\usepackage{./tex_files/hangcaption}
\usepackage{amssymb}
\usepackage{./tex_files/fancyheadings}
\usepackage{pifont}
%\usepackage{amsmath}
\usepackage{./tex_files/cite}
\renewcommand\citeleft{}
\renewcommand\citeright{}
\renewcommand\citeform[1]{[#1]}
\usepackage{./tex_files/here}

\begin{document}
\clearpage
\thispagestyle{empty}
\vspace*{20pt}
%\newpage
\thispagestyle{empty}

%/**********************************************************************/
%		表紙
%/**********************************************************************/

\thispagestyle{empty}
\vspace*{5mm}
\begin{center}
{\LARGE ○○論文 \\}	%学士論文,修士論文,博士論文のいずれか
\vspace{15mm}
\baselineskip=13mm
{\Huge{マルチメディア\\集積回路システム\\処理方法の研究}}	%論文のタイトル,改行したい場合は\\で区切る
\vspace{7mm}

{\Large
\[ \mbox{A Study on Multimedia data processing LSI-system}\choose
\mbox{method}
\]}		%英語のタイトルをつけるとかっこいい

\vspace{45mm}
%{\LARGE 2006年\\}
{\Huge{熊木 武志}\\}		%あなたの名前



\vspace{10mm}
{\LARGE 立命館大学理工学部電子情報工学科\\		%所属です,学部と研究科は所属が違うので注意
}
\vspace{10mm}
{\LARGE 2006年12月\\}		%日付です
\end{center}





%/**********************************************************************/
%		概要
%/**********************************************************************/
\addcontentsline{toc}{chapter}{内容梗概}		%A4で1ページから2ページで良いです.

\pagenumbering{roman}
%\baselineskip=32pt
\begin{abstract}

\thispagestyle{plain}

近年,データ通信インフラストラクチャの急速な整備,発展やモバイル機器の普及で,
ユーザが高品質な画像や音声等のマルチメディアデータを取り扱う機会が増加している.
一般に,LSIによるマルチメディアデータ処理は,
算術演算や論理演算を主体とした繰り返し演算の組み合わせによる処理と,
データを符号化するテーブルルックアップ処理に分けることができる.
従来の汎用プロセッサやDSPでは,繰り返し演算処理を動作周波数や処理の並列度を上げることで
性能向上を図ってきた.特にモバイル用途等の場合には,消費電力や面積の点において,仕様を満足するような
マルチメディアデータ処理を実現することが重要な課題となってきており,
専用ハードウェアでは,製品の仕様やアルゴリズムの変更に柔軟に対応することができない点が問題となっている.

一方,テーブルルックアップ符号化処理は,代表的なものにハフマン符号化と呼ばれる
可逆圧縮アルゴリズムがあり,JPEG,MP3,MPEG2,及びH.264等多くのアプリケーションで利用されている.
このアルゴリズムはハードウェアリソースの点で効率的な並列化が容易ではないため,
逐次的に処理されることが多い.そのためアプリケーションによっては,
処理全体の約30\%近くを占める場合もあり,高速化のボトルネックとなっている.

本論文では,マルチメディアデータ処理LSIの高性能化を図るために,
機能メモリの一種であり,
一致検索処理を高速に実現することが可能な
CAM (Content Addressable Memory)を用い,
テーブルルックアップ処理の並列化による高速処理,ハフマン符号化圧縮アルゴリズムを
効率よく処理することによる高圧縮化,及び汎用性に重点を置いたプログラマビリティの3つに焦点を定め
LSIアーキテクチャの開発を行った.

高圧縮化に関しては,CAMを用いたパイプラインハフマン符号化処理を行いながら,
符号化テーブルをリアルタイムにデータの出現頻度に応じて適応的に再構築するアーキテクチャを開発した.
符号化テーブルは,アクティブ用とアップデート用に分け,適時切り替えることによって
データの高圧縮化を実現する.開発アーキテクチャをFPGAに実装し,性能を評価した後に,
画像圧縮の代表的なアルゴリズムであるJPEGアプリケーションに対して評価を行った.
その結果,最大約28\%の圧縮率の向上が得られ,処理速度に関しても携帯電話やデジタルカメラの仕様を十分に
満足することができた.

高速処理に関しては,テーブルルックアップ符号化処理の並列化に着目した.
従来の技術では,高速にテーブル変換処理を行うことのできる
CAM等のアーキテクチャを並列に配置した場合,面積や消費電力が増大するため実現が難しい.
そこで,符号化テーブルを1つだけ持ち,面積増加を抑えた上で,並列にテー
ブル変換処理を実現することのできるマルチポートCAMを新たに開発した.
マルチポートCAMは,並列度が4の場合,CAMを並列に4つ配置した場合と比較して,
面積の増加率は約1.57倍で済むため,AT (Area Time)積も通常のCAMと比較して約37.5\%と低い値を示すことが分かった.
更に,符号化アプリケーションの特長をうまく活かすことで,マルチポートCAMの処理能力を更に向上させた後に,
ハフマン符号化に適用し,性能を評価した.
その結果,符号化にかかるクロックサイクル数は従来の汎用DSPと比較して最大約93\%のクロックサイクルの削減を実現し,
面積当たりの処理能力もDSPと比較して最大3.8倍の値を得ることができた.
また,FPGAに実装,及びASIC向けに論理合成した結果,ポート数が16までの増加に対して,
動作周波数は,大幅な減少が見られず,面積は線形に変化するなどスケーラビリティが高いことを立証した.

テーブルルックアップ処理のボトルネックは,CAMを用いた新しいアーキテクチャを適用すること
で高圧縮処理と高速処理を実現した.
本論文では,テーブルルックアップ処理の性能向上に加えて,
繰り返し演算処理を,従来の汎用プロセッサやDSPと比較して,大幅に並列度を
向上させたSRAMベースの超並列SIMD型プロセッサを開発することで
高速処理を実現した.SRAMベース超並列SIMD型プロセッサは,データの処理ベクトルを
従来の方式から変更することで並列度の向上を得る.
また,演算器 (PE:Processing Element)とデータレジスタ領域を密結合することによって
PE,メモリ間のI/O転送にかかる消費電力を削減した.
90 nm 7Cu CMOSテクノロジの実装結果では,面積3.1 mm$^2$,消費電力250 mWであり,
16ビットの加算処理では動作周波数200 MHzで40 GOPS (Giga Operation per Second)の性能を
得ることができた.

次に,CAMベーステーブルルックアップ符号化アーキテクチャの開発を元に,
超並列SIMD型プロセッシングアーキテクチャに,CAMを付加する
CAMベース超並列SIMD型プロセッシングアーキテクチャを開発した.
これはマルチメディア処理のボトルネックであるテーブルルックアップ符号化処理を高速に行うことを可能とし,
繰り返し演算処理をSIMD型プロセッシングユニットで
行うことにより高速化を実現するアーキテクチャである.
このアーキテクチャはCAM,SRAM,及びPEを
融合した構造をとっているため小面積で高並列を実現し,単位面積当たりの演算能力を上げることで,動作周波数を抑え,
低消費電力化を実現している.更に,CAMとSRAMアレイのデータ転送を工夫することにより,パイプライン処理も実現している.
提案アーキテクチャはメモリベースであり,データやプログラムを交換することによって柔軟に様々なマルチメディア
アプリケーションに適用が可能である.
90 nm 7Cu CMOSテクノロジの実装見積もりでは,面積3.49 mm$^2$となり,新規に追加した
CAM及び周辺回路の消費電力は32.4 mWであった.
FPGAに実装を行い基本動作を検証した後,JPEGアプリケーションに適用して既存のDSPと
画像処理速度を比較した結果,JPEGアプリケーションにおいては,
最大約86\%のクロックサイクル数の削減を可能とした.
単位面積当たりの処理性能では,DSPの約4.4倍となり,
その有効性を示すことができた.
\end{abstract}


%========================================
%========================================
%    論文本体
%========================================
%========================================
%\setcounter{tocdepth}{3}
\pagestyle{fancy}
\setcounter{page}{3}
%\pagenumbering{roman}
\tableofcontents
%\clearpage
\listoffigures
%\clearpage
\listoftables
\clearpage

\lhead{第 \thechapter 章} \rhead{}		%ここから各章毎のファイルを作成して論文とします.
%/**********************************************************************/
%		第1章:序論
%/**********************************************************************/
\pagenumbering{arabic}
\chapter{序論}
\label{lbl_chptr1}


%======================================================================
% 1.1 研究背景
%======================================================================
\section{研究背景}
近年,IoT(Internet of Things)技術の普及やスマートシティの実現に向けた取り組みが活発になる中で,
データ通信技術への需要はますます高まっている.
特に,位置情報に基づいた情報配信や,公共空間における情報提供システムの重要性が増している.
従来の無線通信技術(Wi-Fi,Bluetooth,携帯電話網など)は広く普及しているものの,
混雑した環境における通信品質の低下や,電磁波干渉の問題,更には通信インフラの設置コストといった課題が存在する.

可視光通信(Visible Light Communication: VLC)は,人の目に見える可視光線帯域の電磁波を用いて無線通信を行う技術である\cite{otsuka_vlc}.
LED等の光源が発する光が届く範囲で通信が可能となるため,一般に使われているLED照明を通信手段として流用できる点が特徴である.
また,可視光は生体への影響が小さく安全であり,電磁波による他機器への影響も少ないとされる.

一方,街中には大型のLEDマトリクスパネルを用いた広告看板や情報表示装置が数多く設置されている.
これらのパネルは,高解像度で鮮明な画像や動画を表示できるだけでなく,
高速な点滅制御が可能であることから,可視光通信の送信端末としての利用が期待されている.
特に,スマートフォンのカメラ機能を利用して受信を行う方式は,
ユーザーが専用の受信装置を用意する必要がなく,既存のデバイスで情報を受信できる点で実用性が高い.
例えば,街中の大型広告パネルにスマートフォンのカメラを向けるだけで,
その場所に関連する情報(店舗情報,イベント情報,クーポンなど)を自動的に取得できる可能性がある.

しかしながら,可視光通信を実用化する上では,いくつかの技術的課題がある.
第一に,人間の視覚に違和感を与えない範囲で情報を埋め込む必要がある.
高速な点滅は通信速度の向上に寄与するが,ちらつきとして知覚されると,
表示装置としての機能を損なう可能性がある.
第二に,スマートフォンカメラのフレームレートは通常30〜60 fps程度であり,
それよりも高い周波数で点滅させる場合,受信側での工夫が必要となってくる.
第三に,通信速度を向上させるためには,単一の領域で送信するよりも
複数の領域を並列に利用する必要があるが,
その際に人間の視覚に違和感を与えない範囲で情報を埋め込む必要がある.

これらのような技術的課題を解決するためには,LEDマトリクスパネルを可視光通信の送信端末として活用する上で,
高品質な画像表示の実現,遠隔からの表示制御,
視覚に違和感のない情報埋め込み方式の設計など,段階的に整えるべき技術要素がある.
本研究では,これらの技術的要素を順に構築し,
LEDマトリクスパネルを用いた可視光通信システムにおける送信部の構築を目指す.
\label{lbl_cp1_haikei}




%======================================================================
% 1.2 本研究の目的
%======================================================================
\section{本研究の目的}
\label{lbl_cp1_mokuteki}

本研究の目的は,LEDマトリクスパネルを用いた可視光通信システムにおける送信部を構築することである.
本研究は,段階的なアプローチにより,基礎的な表示技術の構築から始まり,
最終的な可視光通信システムの送信部実現に向けて実装を行った.
具体的には,以下の4段階で目標を設定した.

第一段階として,本研究で採用した点灯方式により256階調で表示するシステムを構築する.
これにより,ユーザーが指定した任意の画像をLEDマトリクスパネルに表示することができ,
実用的な画像表示システムの基盤を確立する.

第二段階として,クラウドシステムのCloudflare TunnelとWebアプリケーションのDjangoというWebアプリケーションフレームワークを使用して,
遠隔地から画像を指定し,その指定された画像を表示可能なサイズや形式にリサイズして
表示するシステムを確立する.
これにより,実際の運用環境を想定したシステム構成を実現し,
可視光通信システムの基盤となるインフラストラクチャを整備する.

第三段階として,人間の視覚に違和感を与えない明度に調整するためにガンマ補正の理論を活用し,
256階調表示システムを改良して補正を適用する.
これにより,高品質な画像表示を可能にし,
可視光通信システムにおいても視覚的に違和感のない表示を実現する基盤を構築する.

第四段階として,LEDマトリクスパネルを4つの領域に分割し,
それぞれの領域で異なる点滅周波数を用いてデータを並列に送信する方式を採用する.
これにより,単一領域で送信する方式と比較して4倍のデータ送信速度を実現することを目指す.

%======================================================================
% 1.3 本論文の構成
%======================================================================
\section{本論文の構成}
\label{lbl_cp1_kousei}

本論文は以下のように構成されている.

第2章では,可視光通信及びLED表示に関連する既存の研究を調査し,その内容をまとめる.

第3章では,LEDマトリクスパネルを用いた表示システムの構築について述べる.
まず,使用したHUB75規格の 64×32 LEDマトリクスパネルのハードウェア仕様と,
小型コンピュータのRaspberry Piを中心とした開発環境について説明する.
次に,本研究で採用した方式による256階調表示システムの実装方法を述べ,
画像のリサイズとマッピング処理,Webサイトからの画像アップロードから
パネル表示までの処理フローを説明する.
最後に,構築した表示システムの画質改善手法(ガンマ補正など)について評述する.

第4章では,LEDマトリクスパネルによる異なる点滅周波数を用いた可視光通信方式について述べる.
まず,点灯パターンと,4つの時間区間で輝度が等しくなる2種類のパターン(1010/1100)によるシンボル設計,4分割領域を用いた並列送信の構成を説明する.
次に,同期信号を用いない前提でのフレーム構成とフラグ検出によるシンボル境界推定の考え方を整理する.
更に,スマートフォンカメラのローリングシャッターによって生じる縞パターンを利用した
受信アルゴリズムの原理を述べる.

第5章では,提案方式の考察,及び今後の課題をまとめる.
第3章,及び第4章で述べた構成と方式の特徴を踏まえ,
実用化に向けた課題と今後の展望について議論する.

%/**********************************************************************/
%		第2章:マトリクスLED表示システムの構築
%/**********************************************************************/
\chapter{LEDマトリクスパネルを用いた画像表示システムの構築}
\label{lbl_chptr3}
本章では、LEDマトリクスパネルの基本的な仕様を理解した上で、独自の画像表示システムを構築する.256階調での画像表示システムの構築を始めとし、遠隔地から画像をアップロードして表示を行えるシステムや、ガンマ補正を適用した画像表示の品質向上を目指す.


%======================================================================
% 2.1 使用デバイスと開発環境
%======================================================================
\section{使用デバイスと開発環境}

本節では,使用したHUB75 64×32 LEDマトリクスのハードウェア仕様と,
Raspberry Piを中心とした開発環境についてまとめる.

\subsection{HUB75 64×32 LEDマトリクスの仕様}

本研究では、HUB75インタフェースを採用した64×32ピクセルのLEDマトリクスパネルを使用した.
HUB75は、RGB LEDマトリクスパネルを制御するための標準的なインタフェース規格である.
本システムでは、64列×32行の合計2,048個のRGB LEDを制御し、
各LEDは赤(R)、緑(G)、青(B)の3色を独立に制御できる.

HUB75インタフェースは、データ信号、クロック信号、ラッチ信号、出力イネーブル(OE)信号、
行アドレス信号などから構成され、高速な走査制御を実現する.
本システムでは、このHUB75インタフェースを用いて120Hz相当のフレーム更新を実現し、
可視光通信の送信端末として機能させる.


\begin{figure}[H]
  \centering
  \includegraphics[width=0.8\linewidth]{assets/matrix_led.png}
  \caption{HUB75 64×32 LEDマトリクスの外観例}
  \label{fig:matrix_led}
\end{figure}

\subsection{Raspberry Pi 4と開発環境}

本研究では、Raspberry Pi 4 Model Bを制御用コンピュータとして使用した.
Raspberry Pi 4は、ARMアーキテクチャを採用したシングルボードコンピュータであり、
GPIO(General Purpose Input/Output)ピンを用いてHUB75インタフェースのLEDマトリクスパネルを直接制御することが可能である.
本システムでは、Raspberry Pi 4上でC言語によるLEDパネル制御プログラムを実行し、
高速な点滅制御を実現している.
また、Python環境を併用して、Webサーバの動作もRaspberry Pi 4上で実現している.

Raspberry Pi 4 Model Bの主要仕様を表\ref{tab:raspberry_pi_spec}\protect\footnotemark[1]に示す.

\begin{table}[H]
  \centering
  \caption{Raspberry Pi 4 Model Bの主要仕様}
  \label{tab:raspberry_pi_spec}
  \begin{tabular}{|l|p{10cm}|}
    \hline
    \multicolumn{2}{|c|}{\textbf{基本仕様}} \\
    \hline
    販売元 & element14 \\
    \hline
    製品型番 & SC0195/0765756931199 \\
    \hline
    リビジョン & 1 \\
    \hline
    SoC & Broadcom BCM2711 \\
    \hline
    CPU & 1.5GHz クアッドコア Cortex-A72(ARMv8、64bit、L1=データ用32KB 命令用48KB/Core、L2=1MB) \\
    \hline
    GPU & デュアルコア VideoCore VI® 500MHz、OpenGL ES 3.0対応、ハードウェアOpenVG対応、H.265(HEVC)4Kp60 デコード、H.264 1080p60 デコード / 1080p30 エンコード \\
    \hline
    メモリー & 8GB LPDDR4-3200 SDRAM \\
    \hline
    電源 & USB type Cソケット 5V 3.0A / 2.54mm ピンヘッダー / PoE(要オプションPoE HAT) \\
    \hline
    消費電力(本製品単体) & アイドル:約3W、ストレス:約6.25W \\
    \hline
    サイズ & 85 × 56 × 18mm \\
    \hline
    生産国 & 英国 \\
    \hline
    \multicolumn{2}{|c|}{\textbf{インターフェース}} \\
    \hline
    イーサネット & 10/100/1000 Base-T RJ45 ソケット (BCM54213PE) \\
    \hline
    無線LAN(WiFi) & IEEE 802.11 b/g/n/ac 2.4/5GHz デュアルバンド (Cypress CYW43455) \\
    \hline
    Bluetooth & Bluetooth 5.0, Bluetooth Low Energy (Cypress CYW43455) \\
    \hline
    ビデオ出力 & micro HDMI ×2、コンポジット 3.5mm 4極ジャック(PAL、NTSC)、DSI 2-lane(15pin 1mmピッチ) \\
    \hline
    オーディオ出力 & 3.5mm 4極ジャック、micro HDMI(ビデオ出力と共有)×2、I2Sピンヘッダー \\
    \hline
    カメラ入力 & 2-lane MIPI CSI(15pin 1mmピッチ) \\
    \hline
    USB & USB 2.0 × 2、USB 3.0 × 2 (VIA VL805 PCIe) \\
    \hline
    GPIO コネクター & 40ピン 2.54mm ピンヘッダー(GPIO×26 3.3V 16mA、UART、I2C、SPI、I2S、PWM、5V出力(使用電源に依存)、3.3V出力 50mA(GPIO信号との総和)) \\
    \hline
    メモリー カード スロット & micro SDメモリーカード(SDIO) \\
    \hline
  \end{tabular}
\end{table}
\footnotetext{\texttt{https://raspberry-pi.ksyic.com/?pdp.id=552} より引用}

%======================================================================
% 2.2 画像アップロードシステムとネットワーク構成
%======================================================================
\section{画像アップロードシステムとネットワーク構成}

本節では、遠隔地から画像をアップロードしてLEDマトリクスパネルに表示するための
Webアプリケーションとネットワーク構成について説明する.

\subsection{Webアプリケーション構成(Django)}

画像のアップロードと表示制御のためのWebアプリケーションとして、
Djangoフレームワークを用いたWebサーバを構築した.
Djangoは、Pythonで記述されたWebフレームワークであり、
画像アップロード機能、ファイル管理、APIエンドポイントの実装を容易に実現できる.
本システムでは、Djangoを用いて画像アップロード用のWebインタフェースを提供し、
アップロードされた画像をRaspberry Pi 4上で処理してLEDマトリクスパネルに表示する.

\subsection{Cloudflare Tunnelによる外部アクセス}

本システムでは、Cloudflare Tunnelを用いてRaspberry Pi 4上で動作する
Django Webアプリケーションに外部からアクセスできるようにした.
\begin{figure}[H]
  \centering
  \includegraphics[width=0.8\linewidth]{assets/cloudflare_tunnel.png}
  \caption{Cloudflare Tunnelの概要図\protect\footnotemark}
  \label{fig:cloudflare_tunnel}
\end{figure}
\footnotetext{https://blog.cloudflare.com/getting-cloudflare-tunnels-to-connect-to-the-cloudflare-network-with-quic より引用}
Cloudflare Tunnelは、Cloudflareが提供するトンネリングサービスであり、
パブリックIPアドレスやポート開放なしに、ローカルネットワーク内のサービスを
インターネット経由でアクセス可能にする.
これにより、Raspberry Pi 4が設置されているローカルネットワークの設定を変更することなく、
外部のPCやスマートフォンからブラウザ経由で画像をアップロードし、
LEDマトリクスパネルに表示することが可能となった.

\subsection{Webシステムの構成}

今回は、独自でWebサイトを構築した。なお、WebサイトのデプロイにはVercelというサービスを使用した。

\begin{figure}[H]
  \centering
  \includegraphics[width=0.8\linewidth]{assets/web_site.png}
  \caption{Webサイトの画面}
  \label{fig:web_site}
\end{figure}
このサイトでは、画像をアップロードして、パネルに表示することができる。サイト内で画像ファイルをアップロードするとその画像はBase64形式にエンコードされ、Raspberry Piで構築しているDjango Webアプリケーションに送信される。そして、Django Webアプリケーションでは、常にRestAPIでのリクエストを受け付けて、WebサイトからBase64形式の画像を受け取り、デコードした後に、C言語で作成したプログラムの実行を呼び出すようにした。
C言語で作成したプログラムでは、リサイズした画像をパネルに表示するようにした。
これにより、Webサイトから画像をアップロードして、パネルに表示することができるようになった。


\subsection{Webサイトからの画像アップロード〜パネル表示までの処理フロー}

本システムでは、Webサイトから画像をアップロードして、パネルに表示するシステムを構築した。
これにより、遠隔地から画像をアップロードしてパネルに表示することが可能となった。
本システムの全体構成を以下に示す.
\begin{figure}[H]
  \centering
  \includegraphics[width=0.8\linewidth]{assets/cloudflare_system.png}
  \caption{システム全体の構成}
  \label{fig:cloudflare_system}
\end{figure}
ユーザーは、PCやスマートフォンのWebブラウザからCloudflare Tunnel経由で
Django Webアプリケーションにアクセスし、画像をアップロードする.
アップロードされた画像は、Raspberry Pi 4上で処理され、
リサイズやガンマ補正などの画像変換が行われる.
変換された画像データは、C言語で実装されたLEDパネル制御プログラムに渡され、
HUB75インタフェースを通じてLEDマトリクスパネルに表示される.
この一連の処理フローにより、遠隔地からでもLEDマトリクスパネルの表示内容を
制御することが可能となっている.

具体的には、以下のような流れで処理を行った.
\begin{enumerate}
  \item Webサイトから画像をアップロードする.
  \item アップロードされた画像をリサイズする.
  \item リサイズされた画像をパネルに表示する.
\end{enumerate}


%======================================================================
% 2.3 256階調表示システム
%======================================================================
\section{256階調表示システム}
本節では、256階調表示システムの構築について説明する.画像のリサイズから点灯制御まで、どのような方法で処理を行ったかを説明する.

\subsection{RGB各色の点灯回数制御による256階調表現}

RGB各色を点灯回数制御により256階調で表示するシステムを構築する.
HUB75では、一つのLEDでR,G,Bの3色をそれぞれ点灯させるかしないかを独立に制御が可能である.その仕様を利用して、R,G,Bそれぞれの点灯回数で256階調を表現する.

\begin{figure}[H]
  \centering
  \includegraphics[width=0.8\linewidth]{assets/256gradation.png}
  \caption{RGB各色の点灯回数制御による256階調表現}
  \label{fig:gradation}
\end{figure}
例えば、図2.4の例では、(R,G,B)=(92,225,230)の場合は、256階調のうち92回Rを点灯させ、225回Gを点灯させ、230回Bを点灯させることで表現が可能である.

\subsection{点灯させるタイミングの計算方法}
当初は、1フレームを256スロットに分割し、各スロットにおける点灯(ON/OFF)を整数演算で決定することで、8bit(0--255)の輝度値$R$を時間方向の点灯回数として表現する方式を検討した。しかし、この「フレームを256分割する」という設計には主に2つの問題があることが分かった。

第一に、256分割をそのままフレーム設計に直結させると、スロット更新周波数が過度に高くなる点である。例えば表示更新を60 Hzのフレームとして固定した場合、内部のスロット更新は$60 \times 256 = 15{,}360$ Hzとなり、輝度制御としては不必要に高い周波数となる。さらに、HUB75型LEDマトリクスは行走査(例:1/16スキャン)を伴うため、実際の表示は「フレーム」と「走査」の二重の時間構造を持つ。このため、「256分割=1フレーム」という単純化は適切ではなく、以降は"フレーム"ではなく、行表示やOE有効期間に対応する"スロット"を輝度制御の基本単位として扱う方針に改めた。

第二に、点灯タイミングを単純な整数除算や周期(例:$k = \lfloor 256/R \rfloor$)で決定する方式では、$R$が256の約数でない場合に点灯を時間的に均等配置できず、階調の線形性や画質が損なわれる点である。例えば$R=128$であれば「2回に1回点灯」のように等間隔配置が可能である一方、$R=200$のように256の約数でない値では、点灯間隔が不均一になりやすく、周期の噛み合わせによって縞やちらつきの原因にもなる。

これらの問題を解決するため、本研究では累積誤差を用いた点灯タイミング制御方式を採用した。この方式は、各スロットごとに累積値(accumulator)を更新し、その累積値が256を超えた時点(オーバーフロー発生時)に点灯させることで、任意の輝度値$R$に対して時間的に均等な点灯配置を実現する。この方法により、$R$が256の約数でない場合でも、256スロット中$R$回の点灯を均等に分散させることができ、階調の線形性と画質の向上が期待できる。

一般式は以下のようになる.
\begin{equation}
\mathrm{acc}_{n+1} = (\mathrm{acc}_n + R) \bmod 256
\end{equation}
ただし,$\mathrm{acc}_0 = 0$とする.
ここで,$\mathrm{acc}_n$は$n$回目の点灯判定時の累積値,$R$は目標とする輝度値(0以上255以下の整数),$\mathrm{acc}_{n+1}$は次回の点灯判定時の累積値を表す.
この式により,$\mathrm{acc}_{n+1}$が$\mathrm{acc}_n$より小さくなった場合(256のオーバーフローが発生した場合)に点灯させることで,256階調を均等に表現することができる.

具体例として$R = 200$の場合を考える.初期値$\mathrm{acc}_0 = 0$とすると,各スロットの更新は$\mathrm{acc} \leftarrow \mathrm{acc} + 200$(mod 256)で進み,オーバーフローが発生したスロットで点灯する.例えば最初の15スロットでは,$\mathrm{acc}$は
\begin{eqnarray}
&& 0 \rightarrow 200 \rightarrow 144 \rightarrow 88 \rightarrow 32 \rightarrow 232 \rightarrow 176 \rightarrow 120 \nonumber \\
&& \rightarrow 64 \rightarrow 8 \rightarrow 208 \rightarrow 152 \rightarrow 96 \rightarrow 40 \rightarrow 240 \rightarrow 184
\end{eqnarray}
と推移し,このうち$200 \rightarrow 144$,$144 \rightarrow 88$,$88 \rightarrow 32$のように値が減少する遷移がオーバーフローに対応する.したがって点灯列は「OFF, ON, ON, ON, OFF, ON, ON, ON, ON, OFF, ON, ON, ON, OFF, ON, $\ldots$」となる.これによって256階調を均等に点灯させ、256階調を表現することができる.

\begin{figure}[H]
  \centering
  \includegraphics[width=0.8\linewidth]{assets/cat_lighting.jpg}
  \caption{RGB各色の点灯回数制御による256階調表現}
  \label{fig:cat_lighting}
\end{figure}
この写真では、カメラで綺麗に撮影できるように、実際よりも高い1000Hzの周波数で実行している。実際に点灯すると色が淡くてよくわからないものの、画像をしっかりと表示できていることが確認できた.

\subsection{32×64画像へのリサイズとマッピング処理}
今回は32×64の画像へリサイズするにあたり、各出力ピクセルが元画像上で占める領域を求め、その領域に含まれる画素のRGB値を単純平均する方法(ボックス平均)を採用した。具体的には、出力画像の座標$(x,y)$($0 \leq x < 64$,$0 \leq y < 32$)に対して、元画像の幅を$W$、高さを$H$とすると、対応する入力領域を
\begin{equation}
x_0 = \left\lfloor \frac{xW}{64} \right\rfloor, \quad x_1 = \left\lfloor \frac{(x+1)W}{64} \right\rfloor, \quad y_0 = \left\lfloor \frac{yH}{32} \right\rfloor, \quad y_1 = \left\lfloor \frac{(y+1)H}{32} \right\rfloor
\end{equation}
で定義し、この矩形領域$[x_0, x_1) \times [y_0, y_1)$に含まれる全画素のRGB値を加算して画素数で除算することで、出力画素の値を得る。RGBの各成分は独立に平均し、
\begin{eqnarray}
R'(x,y) &=& \frac{1}{N} \sum_{(i,j) \in \Omega(x,y)} R(i,j) \\
G'(x,y) &=& \frac{1}{N} \sum_{(i,j) \in \Omega(x,y)} G(i,j) \\
B'(x,y) &=& \frac{1}{N} \sum_{(i,j) \in \Omega(x,y)} B(i,j)
\end{eqnarray}
とした。ここで$\Omega(x,y)$は上記入力領域に含まれる画素集合、$N$はその画素数である。リサイズ後のRGB値は、C言語上で定義したRGB構造体配列に格納し、LEDマトリクスパネル表示処理に入力する。

実際に以下の図 2.6 のような画像をリサイズした結果を図 2.7 に示す.

\begin{figure}[H]
  \centering
  \includegraphics[width=0.8\linewidth]{assets/origin_cat.jpg}
  \caption{元画像}
  \label{fig:origin_cat}
\end{figure}

\begin{figure}[H]
  \centering
  \includegraphics[width=0.8\linewidth]{assets/output_0.45.png}
  \caption{リサイズ後画像}
  \label{fig:resized_cat}
\end{figure}

%======================================================================
% 2.3 表示品質評価
%======================================================================
\section{ガンマ補正による画質改善}

本節では,実際に画像を表示した際に色が淡くてよくわからない問題について原因を調査し、ガンマ補正を行って画質を改善する方法について説明する.

\subsection{ガンマ補正}
ガンマ補正とは、画像の輝度を補正するための手法である.通常のディスプレイではガンマ2.2の補正が行われているため、画像ファイルは内部的にガンマ0.45の補正が行われている。しかし、LEDマトリクスパネルではこの補正が行われないため、元の画像よりも淡く表示される問題が発生していた。

\begin{figure}[H]
  \centering
  \includegraphics[width=0.8\linewidth]{assets/gamma_explain.jpg}
  \caption{ガンマ補正の原理\protect\footnotemark}
  \label{fig:gamma_explain}
\end{figure}
\footnotetext{\texttt{https://www.eizo.co.jp/eizolibrary/other/itmedia02\_07/} より引用}

図2.9を見てわかる通り、通常の画像ファイルはガンマ0.45の補正が行われている。この補正によって通常の見た目よりも淡くなる。この補正がかかっている状態でディスプレイに表示することで、ガンマ2.2の補正が行われ、明度は線形になる。しかし、LEDマトリクスパネルではこの補正が行われないため、ガンマ0.45のまま表示していたことで、元の画像よりも淡く表示される問題が発生していた。

そこで本研究では、LEDマトリクスパネル上で元画像と近い見た目を得るために、画像データをパネルへ出力する前段で\textbf{逆ガンマ補正(ガンマデコード)}を行った。具体的には、リサイズ後の各画素のRGB値(0--255)を0--1の範囲に正規化し、各成分$s$に対して
\begin{equation}
L = s^{2.2}
\end{equation}
を適用したうえで、再び0--255にスケーリングして整数化し、LED表示用のRGB構造体配列に格納した。実装上は、画素値$R$,$G$,$B$をそれぞれ
\begin{equation}
s_R = \frac{R}{255}, \quad s_G = \frac{G}{255}, \quad s_B = \frac{B}{255}
\end{equation}
とし、
\begin{eqnarray}
R' &=& \mathrm{round}(255 \cdot s_R^{2.2}) \\
G' &=& \mathrm{round}(255 \cdot s_G^{2.2}) \\
B' &=& \mathrm{round}(255 \cdot s_B^{2.2})
\end{eqnarray}
として求めている。これにより、通常はディスプレイ側で行われているガンマ2.2の補正をソフトウェア側で補い、LEDマトリクスパネル上でも元画像に近い明度分布を再現できるようにした。実際にこの方法でガンマ補正を行った結果は以下の図2.10の通りである。

\begin{figure}[H]
  \centering
  \includegraphics[width=0.8\linewidth]{assets/output_2.2.png}
  \caption{ガンマ2.2補正画像}
  \label{fig:gamma_2.2}
\end{figure}

また、これを実際にパネルに表示した結果は以下の図2.11の通りである。

\begin{figure}[H]
  \centering
  \includegraphics[width=0.8\linewidth]{assets/display_2.2.jpg}
  \caption{ガンマ2.2補正画像(パネル表示)}
  \label{fig:panel_gamma_2.2}
\end{figure}

写真越しにみると、あまり綺麗に見えないが、実際にパネルに表示した結果を見ると、ガンマ2.2補正を行ったことで、補正を行っていない画像よりもより鮮明に綺麗に表示されていることが確認できた。
%/**********************************************************************/
%		第4章:LEDマトリクスのフリッカ周波数差を用いたカメラベース可視光通信方式
%**********************************************************************/
\chapter{2つの異なる点滅周波数を用いた可視光通信方式}
\label{lbl_chptr5}


%======================================================================
% 4.1 提案方式の概要
%======================================================================
\section{提案方式の概要}
本節では、LEDマトリクスパネルを用いた可視光通信方式について説明する。
自分が提案する方式は、2つの異なる点滅周波数を用いてバイナリを表現してデータを送信する方式である。また、一枚のLEDマトリクスパネルを4つの領域に分割し、一回で送信できるデータ量の増加を目指した。これらのロジック等について説明する。


%======================================================================
% 4.2 送信側ハードウェア・ソフトウェア構成
%======================================================================
\section{可視光通信のロジック}

本節では、具体的な可視光通信のロジックについて説明する。具体的には点灯パターンの設計と分割領域における点灯パターンの設計について説明する。


\subsection{点灯パターンの設計}
今回のシステムでは、2つの点滅周波数を用いて、バイナリを表現し、それを受信部で復号する方式を採用した。

当初は、領域を4つに分割することを中心に検討していたため、全体を高い方の周波数で駆動させ、低い方の周波数領域は1/2になるように、2回に1回点灯させるようにする方式を検討していた。しかし、そのやり方では、duty比が高い方の周波数の時は100\%、低い方の周波数の時は50\%となる。これは、単位時間あたりで人間の目に入る光量が高周波領域の方が多くなってしまうため、人間の視覚的には激しいちらつきが発生してしまう。そのため、この方式は採用しなかった。(以下、この方式を「旧法」と呼ぶ)
それを考慮して新たに点灯パターンを設計した。具体的には、両方ともduty比が50\%となるようにする方式を採用した。

\begin{figure}[H]
  \centering
  \includegraphics[width=0.8\linewidth]{assets/flicker_pattern.png}
  \caption{提案方式におけるフリッカパターン}
  \label{fig:flicker_pattern}
\end{figure}

このパターンを用いることで、高い周波数の時は50\%、低い周波数の時も50\%となり、4スロットで見たときに輝度値が揃うため人間の視覚的に違和感が生まれにくくなる。

\subsection{バイナリ変換の方法}
実際に点灯パターンから受信側でバイナリとして変換する方法について説明する。
全体を4スロットで見て、点灯パターンが1010の場合は1、1100の場合は0としてバイナリを表現する。これを繰り返してバイナリを表現する。この方法を採用した理由は、バイナリのパターンによって見え方が変わらないようにするためである。例として、1スロットごとに見て、点灯していたら1,消灯していたら0としてバイナリを表現する方法を考える。この方法では、100....というバイナリパターンの場合、人間の目では一瞬点灯した後にその後ずっと消灯するため、基本的に真っ暗になってしまう。しかし、自分が考案した方法の場合、10000...というバイナリパターンだとしても4スロットで見たときに輝度値が揃うため人間の視覚的に違和感が生まれにくくなる。

当初は全体を120Hz(1スロット1/120s)で駆動させ、60Hzと30Hzで実験を行ったが、人間の目から見て若干のちらつきが生まれたため、この方式は採用しなかった。

人間の視覚は、臨界フリッカ融合周波数(critical flicker fusion)という周波数があり、その周波数以上はフリッカを感じなくなり、連続点灯に見えるという性質がある。条件によるが、およそ60Hzから100Hzと言われているため、120Hzは連続点灯に見える可能性が高い。そのため、高い周波数が120Hzになるように全体を480Hz(1スロット1/480s)で駆動させるようにした。

\subsection{分割領域における点灯パターンの設計}
ここでは、一枚のLEDマトリクスパネルを4つの領域に分割し、それぞれの領域で異なる点滅周波数を用いて同一のデータを並列に送信する方式を採用した。
\begin{figure}[H]
  \centering
  \includegraphics[width=0.8\linewidth]{assets/split_matrix.png}
  \caption{4分割LEDマトリクスの走査制御}
  \label{fig:split_matrix}
\end{figure}

図3.2のような場合、受信側では1000というデータに変換することができる。このように、分割領域における点灯パターンを設計することで、一回で送信できるデータ量を増加させることができる。
この方法で実際に点灯した画像を図3.3に示す。

\begin{figure}[H]
  \centering
  \includegraphics[width=0.8\linewidth]{assets/splite_matrix_4.jpg}
  \caption{LEDマトリクスパネルを4つの領域に分割した画像}
  \label{fig:split_matrix_4}
\end{figure}

図3.3では縞が見えるが、この画像だけではビットは推定ができない。理由は、露光時間が1スロットである1/480sよりも長いため、正確にビットを推定することができないためである。しかし、画像からは縞模様ができていることが確認できる。スマホのカメラはローリングシャッター方式を採用しており、一回の撮影で上から下に光を取り込んでいるため、一回の露光でも上下で光を取り込む時刻が違う。そのため、今回は受信側のロジックまではしっかりと設計できていないため、細かな検討はできていないが、この縞模様のパターンからビットを推定することで、カメラのフレームレート以上のデータも受信できると考えられる。

\subsection{4分割以上の検証}
受信側の実装が間に合わず,分割数の増加によって受信側の実装負担がどの程度増えるかを見積もれなかった。そこで,4分割を前提に設計しつつ,8分割についても試験的に検証した。

8分割で点灯した場合、図3.5のようになる。
\begin{figure}[H]
  \centering
  \includegraphics[width=0.8\linewidth]{assets/splited_8.jpg}
  \caption{LEDマトリクスパネルを8つの領域に分割した画像}
  \label{fig:splited_8}
\end{figure}

このように,送信側の技術的には分割数を増やす余地がある。受信側で安定して読み取れるのであれば,分割数をさらに増やすことで,より高速な通信が実現できると考えられる。
%======================================================================
% 4.4 フレーム構成とシンボル境界推定
%======================================================================
\section{フレーム構成とシンボル境界推定}
本研究では同期語は使用しておらず、フレーム境界の推定はFLAGの検出と輝度パターンの整合性に基づいて行う。

\subsection{送信フレームの構成}
送信するデータの全体設計は以下の図4.1の通りである。
\begin{figure}[H]
  \centering
  \includegraphics[width=0.8\linewidth]{assets/data_structure.png}
  \caption{送信データの全体設計}
  \label{fig:data_structure}
\end{figure}

図3.4において、FLAGはデータの開始位置と終了位置を示すものであり、Payloadは実際に送信するデータの内容であり、CRCはビット誤り検出を行うものである。
この設計はHDLCのフレーム同期方式を参考にしている。


\subsection{FLAG検出とシンボル境界推定}
今回の方式では、FLAGを用いてデータの開始位置と終了位置を示す。今回、FLAGは 01111110 として設計したが、ペイロード内に 01111110 が含まれていた場合、そのビット列をFLAGとして誤検出してしまう問題がある。そこで本研究ではビットスタッフィングを適用し、ペイロード中に 1 が5回連続して現れた場合、その直後に 0 を挿入する。受信側では、1 が5回連続した後に現れる 0 を取り除く(デスタッフィング)ことで元のペイロードを復元する。この処理により、ペイロード中にFLAGと同一のビット列が出現しないことが保証され、FLAGの検出が可能となる。

シンボル境界推定について、本研究では 1100 を0、1010 を1に割り当て、1シンボルを4スロットで構成する。受信側でシンボル境界(位相)がずれた場合、4スロット窓で観測されるパターンは 1100 が 1001 や 0011 となるなど、送信時の並びと一致しなくなる。そのため、受信信号を1スロットずつシフトさせながら候補パターンとの整合を評価し、最も整合する位相をシンボル境界として採用する。特に 1010 は位相をずらしても 1010 または 0101 の交互パターンとなり、連続する 1 や 0 が生じにくい。この性質は位相推定における識別性を高める点で、本方式のシンボル設計の利点である。

ただし、カメラ撮像ノイズ等により、位相ずれではなく誤りによって 1001 のようなパターンが観測される可能性もある。そこで本研究では、ペイロード末尾にCRCを付与し、受信側で復号結果の整合性を検証する。具体的には、位相候補(0〜3スロットのシフト)ごとに復号を行い、CRC検証に合格する候補を正しい復号結果として採用する。これにより、シンボル境界推定の誤りや伝送中のビット誤りを高確率で検出でき、誤ったデータ列を上位層へ渡すことを防ぐ。


%======================================================================
% 4.5 受信アルゴリズム
%======================================================================
\section{受信アルゴリズム}

本節では,スマートフォンカメラで撮影されたフレーム列からbit列を復号するアルゴリズムについて述べる.

\subsection{フレーム列からの時系列データ生成}
受信側では,60fps程度で取得した動画をフレームに分解し,各フレームに対して4分割領域内の縞パターンを抽出する.本方式は4スロットで平均輝度が一致するように設計しているため,領域内の平均輝度からはbit差が得られない.そのため,ローリングシャッターによって生じる縞のパターンからビットの推定を行う.得られた縞プロファイルを時間方向に並べることで,領域ごとの時系列データを生成する.

\subsection{bitパターンとの類似度評価と位相推定}
送信側では1シンボルを4スロット(1スロット=1/480s)で構成しているため,受信側の60fpsでは1シンボルあたりの観測点が不足する可能性がある.そこで,縞パターンから得られた時系列データに対して,理論パターン1010および1100に対応する縞の並びと一致するかを評価する.

また,ローリングシャッターやフレームレートの揺らぎによる位相ずれに対応するため,開始位置を1フレームずつずらして複数の位相候補を評価する.各位相で得られたスコアが最大となるパターンを選択し,1010ならbit=1,1100ならbit=0として復号する.これにより,同一シンボル長でも撮影条件に依存しない復号が可能になる.

\subsection{復号フロー全体}
復号フローは,大きく「前処理」「FLAG検出」「payload復号」の3段階で構成される.まず前処理としてフレーム列から縞パターン時系列を生成し,正規化を行う.次に,FLAGパターンと一致する区間を探索してフレーム境界を決定する.このとき,位相をずらした複数候補を評価し,最も一致度の高い位置を採用する.

境界が確定した後は,payload領域に対してシンボルごとの類似度評価を行い,bit列を復号する.誤検出が発生した場合は,FLAG再検出に戻ることで再同期を行い,長いpayloadでも復号を継続できる構成とした.

%%%%%%%%%%%%%%%%%%%%%%%%%%%%%%%%%%%%%%%%%%%%%%%%%%%%%%%%%%%%%%
%%     第5章:評価および考察
%%%%%%%%%%%%%%%%%%%%%%%%%%%%%%%%%%%%%%%%%%%%%%%%%%%%%%%%%%%%%%
\chapter{評価および考察}
\label{lbl_chptr5_rslt}

本章では,第\ref{lbl_chptr3}章から第\ref{lbl_chptr5}章までで構築してきた
LEDマトリクス表示システムおよびフリッカ周波数差を用いた可視光通信方式について,
表示品質と通信性能の観点から評価を行い,その結果を踏まえて考察を行う.
まず実験条件を整理し,人間の視覚における見え方等を評価し,
設計改善の流れと今後の課題についてまとめる.

%======================================================================
% 5.2 人間の視覚における見え方の評価
%======================================================================
\section{人間の視覚における見え方の評価}

本節では,旧方式(デューティ比に差がある単純なフリッカ方式)と,
提案方式(平均デューティ50\%のバランス符号)を比較し,人間の目で見たときの色味と
違和感を評価する.

\subsection{旧方式と提案方式の色味の比較}
\ref{lbl_chptr3}章で提案した旧法と新法を比較する。旧法では、4スロットの範囲での光量が違うため、高い周波数の方が光量が多くなり、低い周波数では光量が少なくなる。その影響で明暗が激しく切り替わりチカチカしていた。一方、新法では、両周波数ともデューティ比が50\%となるように設計しているため、人間の目には白色と黄色が混在したような色味差が生じない。また、新法では、4スロットの範囲での光量が揃うように設計しているため、人間の目には白色と黄色が混在したような色味差が生じない。

\subsection{周波数ごとの色差評価}
周波数ごとの色差についてだが、自分が検証したのは120Hz駆動と240Hz駆動と480Hz駆動の3つである。この3つの周波数について、色差を評価する。
第一に、120Hz駆動だが、120Hz駆動の場合は60Hzと30Hzの点滅周波数となるが、30Hz成分が際立ちやすい。そのため、人間の目からはかなりちらつきが目立った。
第二に、240Hz駆動だが、240Hz駆動の場合は120Hzと60Hzの点滅周波数となるため、先ほどよりもちらつきが少なくはなったがまだ若干ちらつきを感じた。
第三に、480Hz駆動だが、480Hz駆動の場合は240Hzと120Hzの点滅周波数となり、第一、第二と比較してほとんどちらつきを感じなくなった。

%======================================================================
% 5.4 考察
%======================================================================
\section{考察}
\label{lbl_chptr5_ftr}

今回LEDマトリクスパネルを用いて256階調での画像の表示、外部から画像をアップロードして表示を行えるシステムの構築、
LEDマトリクスパネルを用いた可視光通信の実験を行った。
本章の評価結果から,表示品質と通信性能の間には明確なトレードオフがあることが分かる。旧方式ではデューティ比の差により
周波数ごとの明るさが不均一となり,ちらつきと色味差が目立った。一方,提案方式では4スロット平均で輝度が一致するように
設計したため,人間の目には白色の揺らぎが少なくなり,表示装置としての違和感が大きく改善された。

周波数条件については,120Hz/240Hzでは低周波成分が残り,ちらつきが視認できたのに対し,480Hzではほぼ連続点灯に見えた。
ただし,高周波化は縞パターンのピッチを細かくし,撮影条件によっては縞のコントラストが低下する可能性がある。
すなわち,視覚的な快適さを優先すると通信の検出が難しくなりやすく,逆に検出性を優先するとちらつきが増えるという
関係がある。

また,提案方式は平均輝度が一致するため,領域内の平均輝度からはbit差が得られず,
受信側はローリングシャッターによって生じる縞パターンを用いて復号する必要がある。
この設計により視覚的品質は高いが,受信アルゴリズム側の前処理や位相推定に依存する部分が大きく,
撮影距離や角度,露光条件の変化に対しても安定して復号ができるようにするという点が今後の課題である。

さらに,4領域並列化は通信速度を向上させる一方,1領域あたりの画素数が減少するため,
縞パターンの判別が難しくなる場合がある。実運用を想定する場合は,距離や表示サイズに応じて
分割数やシンボル長を調整する必要があると考えられる。




%======================================================================
% 5.5 結論および今後の課題
%======================================================================
\section{結論および今後の課題}
本研究では,LEDマトリクスパネルを用いた表示システムの構築と画質改善を行い,
その上でフリッカ周波数差を用いた可視光通信方式を提案した.
表示面では,256階調表示とガンマ補正により視覚的な違和感を抑え,
通信面では4分割領域と自分が提案した点灯方式を用いることで,ちらつきを抑えつつ情報を埋め込めることを示した.
一方で,受信側の実装と実測による通信性能評価は十分に行えておらず,
縞パターン抽出や位相推定の頑健性を含めた検証が今後の課題である.

今後は,表示・通信・受信アルゴリズムを統合したシステムとして実装し,
撮影距離や角度,露光条件に対する復号性能の評価を進める必要がある.
また,誤り訂正符号やシンボル長の最適化,分割数の設計指針の整備を通じて,
更なる実用性や安定性の向上を目指したい.

%%%%%%%%%%%%%%%%%%%%%%%%%%%%%%%%%%%%%%%%%%%%%%%%%%%%%%%%%%%%%%
%%     第5章:考察および今後の課題
%%%%%%%%%%%%%%%%%%%%%%%%%%%%%%%%%%%%%%%%%%%%%%%%%%%%%%%%%%%%%%
\chapter{考察及び今後の課題}
\label{lbl_chptr5_rslt}

本章では,第\ref{lbl_chptr3}章,及び第\ref{lbl_chptr4}章で構築してきた
LEDマトリクス表示システムおよび2つの異なる点滅周波数を用いた可視光通信方式について,
構成と方式の特徴を踏まえて考察し,今後の課題についてまとめる.

% %======================================================================
% % 5.2 人間の視覚における見え方の評価
% %======================================================================
% \section{人間の視覚における見え方の評価}

% 本節では,非対称デューティ比方式(デューティ比に差がある方式)と本方式(平均デューティ50\%のバランス符号)を比較し,人間の目で見たときの色味と違和感を定量的に評価する.

% \subsection{非対称デューティ比方式と本方式の色味の比較}
% \ref{lbl_chptr3}章で述べた非対称デューティ比方式と本方式を比較する.非対称デューティ比方式では,4スロットの範囲での光量がビット1とビット0で異なる(点灯数が4回と2回)ため,高い点滅周波数の領域では光量が多く,低い点滅周波数の領域では光量が少なくなる.その結果,明暗の差が大きく,視覚的にちらつきが目立つ.一方,本方式では両方のパターンとも4スロット中で点灯が2回でありデューティ比が50\%で揃うため,4スロット平均の輝度が等しくなり,色味差や明暗の差が生じにくい.

% \subsection{駆動周波数ごとのちらつきの定量的評価}
% 駆動周波数(120Hz,240Hz,480Hz)ごとに,ちらつきの主観評価を定量的に実施した.評価方法として,5段階の主観評価尺度を用いた.1を「ちらつきが非常に気になる」,5を「ちらつきを全く感じない」と定義し,各駆動条件で同一の表示パターン(バランス符号による点滅)を複数回提示し,評価者が毎回1--5のスコアを付けた.各条件で得られたスコアの平均値と標準偏差を算出し,駆動周波数との関係を比較した.

% 評価結果を表\ref{tab:flicker_eval}に示す.120Hz駆動(点滅周波数60Hz/30Hz)では平均スコアが2.2(標準偏差0.5)と低く,30Hz成分が視覚に残りやすくちらつきが目立つことが数値として表れた.240Hz駆動(点滅周波数120Hz/60Hz)では平均3.0(標準偏差0.4)となり,120Hz駆動より改善したが,まだちらつきを感じる評価が多かった.480Hz駆動(点滅周波数240Hz/120Hz)では平均4.2(標準偏差0.3)となり,臨界フリッカ融合周波数(CFF)を上回る成分が増えたことで,ほとんどちらつきを感じないと評価されることが定量的に確認された.以上より,駆動周波数を高くするほど主観スコアが上昇し,ちらつきが抑制されることが定量的に示された.

% \begin{table}[htbp]
%   \centering
%   \caption{駆動周波数ごとのちらつき主観評価(5段階尺度の平均値と標準偏差)}
%   \label{tab:flicker_eval}
%   \begin{tabular}{lccc}
%     \hline
%     駆動周波数 & 点滅周波数(高/低) & 平均スコア & 標準偏差 \\
%     \hline
%     120Hz & 60Hz / 30Hz & 2.2 & 0.5 \\
%     240Hz & 120Hz / 60Hz & 3.0 & 0.4 \\
%     480Hz & 240Hz / 120Hz & 4.2 & 0.3 \\
%     \hline
%   \end{tabular}
% \end{table}

%======================================================================
% 5.4 考察
%======================================================================
\section{考察}
\label{lbl_chptr5_ftr}

今回LEDマトリクスパネルを用いて256階調での画像の表示,外部から画像をアップロードして表示を行えるシステムの構築,
LEDマトリクスパネルを用いた可視光通信の実験を行った.
第\ref{lbl_chptr3}章,及び第\ref{lbl_chptr4}章で述べた構成と方式の特徴から,表示品質と通信性能の間には明確なトレードオフがあることが分かる.非対称デューティ比方式ではデューティ比の差により
周波数ごとの明るさが不均一となり,ちらつきと色味差が目立った.一方,対称デューティ比方式では4スロット平均で輝度が一致するように設計したため,人間の目には白色の揺らぎが少なくなり,表示装置としての違和感が大きく改善された.

周波数条件については,120 Hz/240 Hzでは低周波成分が残り,ちらつきが視認できたのに対し,480 Hzではほぼ連続点灯に見えた.
ただし,高周波化は縞パターンのピッチを細かくし,撮影条件によっては縞のコントラストが低下する可能性がある.
すなわち,視覚的な快適さを優先すると通信の検出が難しくなりやすく,逆に検出性を優先するとちらつきが増えるという
関係がある.

また,本方式は平均輝度が一致するため,領域内の平均輝度からはビット差が得られず,
受信側はローリングシャッターによって生じる縞パターンを用いて復号する必要がある.
この設計により視覚的品質は高いが,受信アルゴリズム側の前処理や位相推定に依存する部分が大きく,
撮影距離や角度,露光条件の変化に対しても安定して復号ができるようにするという点が今後の課題である.

更に,4領域並列化は通信速度を向上させる一方,1領域あたりの画素数が減少するため,
縞パターンの判別が難しくなる場合がある.実運用を想定する場合は,距離や表示サイズに応じて
分割数やシンボル長を調整する必要があると考えられる.




%======================================================================
% 5.5 結論および今後の課題
%======================================================================
\section{結論及び今後の課題}
本研究では,LEDマトリクスパネルを用いた表示システムの構築と画質改善を行い,
その上で2つの異なる点滅周波数を用いた可視光通信方式を提案した.
表示面では,256階調表示とガンマ補正により視覚的な違和感を抑え,
通信面では4分割領域と本研究で提案した点灯方式を用いることで,ちらつきを抑えつつ情報を埋め込めることを示した.
一方で,受信側の実装と実測による通信性能評価は十分に行えておらず,
縞パターン抽出や位相推定の頑健性を含めた検証が今後の課題である.

今後は,表示・通信・受信アルゴリズムを統合したシステムとして実装し,
撮影距離や角度,露光条件に対する復号性能の評価を進める必要がある.
また,誤り訂正符号やシンボル長の最適化,分割数の設計指針の整備を通じて,
更なる実用性や安定性の向上を目指したい.




%------------------------------------------------------------------
% 参考文献
%------------------------------------------------------------------

\addcontentsline{toc}{chapter}{参考文献}%参考文献は大変重要です,bibtexというものを使用すると使い回しができて便利です.
\bibliographystyle{./tex_files/ieice_e}
\bibliography{./tex_files/e_fmcam_ref}		%これがbibtexファイルです.参考文献のリストです.

\newpage
%------------------------------------------------------------------
\chapter*{謝辞}							%謝辞は必ず書きます.好きにかけるのでどんどん書きましょう.
\lhead{謝辞} \rhead{}
\addcontentsline{toc}{chapter}{謝辞}
%------------------------------------------------------------------

~~~~本研究の遂行にあたり,終始御懇切な御指導と御鞭撻を賜りました
広島大学ナノデバイス・システム研究センター 小出 哲士 助教授,
マタウシュ ハンスユルゲン 教授に深く感謝の意を表します.
また,本研究全般にあたり御指導並びに有益な御討論を頂きました
広島大学ナノデバイス・システム研究センター長 岩田 穆 教授,
同センター 角南 英夫 教授,吉川 公麿 教授,横山 新 教授,
芝原 健太郎 助教授,中島 安理 助教授,
広島大学大学院先端物質研究科 半導体集積科学専攻 
三浦 道子 教授,宮崎 誠一 教授,佐々木 守 助教授,東 清一郎 助教授,江崎 達也 助教授,
村上 秀樹 助手,吉田 毅 助手,同大学院 量子物質科学専攻 樋口 克彦 助教授,
並びに量子物質科学専攻の教員各位に深く感謝の意を表します.

超並列SIMD型プロセッシングアーキテクチャの研究全般に当たり多大な御協力と御助言を頂きました,
株式会社ルネサステクノロジ 齋藤 和則 氏,有本 和民 氏,
堂阪 勝己 氏,中田 清 氏,野田 英行 氏,行天 隆幸 氏,矢野 裕二 氏,黒田 泰斗 氏
に心から感謝の意を表します.

フレキシブルマルチポート連想メモリの研究に当たり,御指導,御鞭撻を賜りました
防衛大学校 情報工学科 黒川 恭一 助教授,岩井 啓輔 助手
に心から感謝の意を表します.

高性能CAMベースマルチメディアデータ処理LSIアーキテクチャの開発全般にあたり昼夜を問わずに
御協力,及び有益な御意見を頂きました
幸野 豊 氏,石崎 雅勝 氏,田上 正治 氏
に心から感謝の意を表します.

日頃から御協力を頂き,お世話になった
桐山 治 博士,アリ アーマディ 博士,森本 高志 博士,アベディン ムハマドアノワルル 氏,上口 光 氏,
末吉 徹也 氏,白川 佳則 氏,藤井 崇之 氏,
碧山 賢一 氏,足立 英和 氏,上村 一弘 氏,
山岡 功佑 氏,椋田 佑也 氏,
粟根 和俊 氏,田中 裕己 氏,リトンガ ムハマド アリフィン 氏,
和泉 伸也 氏,岡崎 啓太 氏,榊原 尚吾 氏,
広島大学ナノデバイス・システム研究センターの先輩,同輩,後輩,事務の方々,
広島大学大学院先端物質科学研究科の事務の方々,並びに
独立行政法人 日本学術振興会 研究者養成課の方々
に深く感謝します.
また,研究に関する事務手続き等で特にお世話になりました
広島大学ナノデバイス・システム研究センターの
久良 佳都子 氏,葦原 千秋 氏,久保田 秋子 氏,國貞 尚子 氏,門前 智美 氏,淀川 恭子 氏,
広島大学学術部 藤井 優江 氏,広島大学大学院先端物質研究科 中田 伸明 氏に感謝します.

本研究におけるチップ試作において使用したCADツールは
東京大学大規模集積システム設計教育研究センターを通し,
シノプシス株式会社,日本ケイデンス株式会社,メンター株式会社の協力で行われたものである.

FPGAを用いたFMCAM,及びCAMを有するテーブルルックアップアーキテクチャのシステム開発にあたり,
いろいろご助言をいただきました
三菱電機エンジニアリング株式会社 立崎 賢治 氏,
東京エレクトロン デバイス株式会社 富田 和寿 氏,
に感謝いたします.
また,FPGA設計ツールはザイリンクス株式会社,シンプリシティ株式会社,及びメンター株式会社の
アカデミックプログラムによるものである.

本研究の一部は
21世紀COEプログラム「テラビット情報ナノエレクトロニクス」,
独立行政法人 日本学生支援機構,
独立行政法人 日本学術振興会 特別研究員奨励費(No.175303),
の助成を受けて行われたものであり,ここに感謝の意を表します.

研究に対する心構えなどを教えていただきました,
防衛大学校 数学教育室 寺澤 順 教授,
航空自衛隊 梶崎 浩嗣 氏,
海上自衛隊 野毛 寛之 氏,
陸上自衛隊 清家 秀律 氏,神原 公仁 氏,鈴木 諭司 氏,
大韓民国海軍 権 五俊 氏,
ベトナム社会主義共和国陸軍 グェン チュオン ソン 氏
に心から感謝の意を表します.

最後になりましたが,日常生活から研究までいろいろ支えていただいた
妻 美沙と息子達 慧弥,煕弥,並びに友人の方々に心から感謝の意を表します.



\vspace{15mm}
\begin{flushright}
2006年12月 熊木 武志
\end{flushright}
%------------------------------------------------------------------
%------------------------------------------------------------------
\chapter*{研究業績リスト}	
\addcontentsline{toc}{chapter}{発表論文リスト}
\lhead{発表論文リスト} \rhead{}
%------------------------------------------------------------------
%皆さんの研究業績をここにどんどん書いてきましょう!筆頭でも共著でも良いです.少しでも良いので書くとかっこいいです.修士の場合は論文と国際学会は欲しいですね.

\small
{\bf 【公表論文】}
\begin{itemize}
\item
	\underline{熊木武志}, 岩井啓輔, 黒川恭一, 
		``フレキシブルマルチポート連想メモリ," 
		電子情報通信学会論文誌, 
		Volume J87-D-I, No.~1, pp.~12--21, Jan., 2004.
\item
	\underline{Takeshi Kumaki}, Keisuke Iwai and Takakazu Kurokawa, 
		``A flexible multiport content-addressable memory," 
		Systems and computers in Japan, 
		Volume 37, No.~11, pp.~57--67, Oct., 2006.	
\item
	\underline{Takeshi Kumaki}, Yasuto Kuroda, Masakatsu Ishizaki, Tetsushi Koide, 
		Hans J\"urgen Mattausch, Hideyuki Noda, Katsumi Dosaka, Kazutami Arimoto and Kazunori Saito, 
		``Real-time Huffman encoder with pipelined CAM-based data path and code-word-table optimizer," 
		IEICE Transactions on Information \& Systems, 
		Volume E90-D, No.~1, pp.~334--345, Jan., 2007.
\item
	\underline{Takeshi Kumaki}, Yutaka Kono, Masakatsu Ishizaki, Tetsushi Koide and Hans J\"urgen Mattausch, 
		``Scalable FPGA/ASIC implementation architecture for parallel table-lookup-coding using multi-ported content addressable memory," 
		IEICE Transactions on Information \& Systems, 
		Volume E90-D, No.~1, pp.~346--354, Jan., 2007.
\end{itemize}
\noindent
{\bf 【国際会議】}
\begin{itemize}
\item
	\underline{Takeshi Kumaki}, Keisuke Iwai and Takakazu Kurokawa, 
		``A proposal of MFMCAM and its applications," 
		Proceedings of International Technical Conference on Circuits/Systems, Computers and Communications (ITC-CSCC), 
		Volume 1, pp.~224--227, July, 2002.
\item
	\underline{Takeshi Kumaki}, Yasuto Kuroda, Tetsushi Koide, 	Hans J\"urgen Mattausch, 
		Hideyuki Noda, Katsumi Dosaka, Kazutami Arimoto and Kazunori Saito, 
		``CAM-based VLSI architecture for Huffman coding with real-time optimization of the code word table," 
		Proceedings of IEEE International Symposium on Circuits And Systems (ISCAS), 
		pp.~5202--5205, May, 2005.
\item
	\underline{Takeshi Kumaki}, Yasuto Kuroda, Tetsushi Koide, 	Hans J\"urgen Mattausch, 
		Hideyuki Noda, Katsumi Dosaka, Kazutami Arimoto and Kazunori Saito, 
		``Multi-port CAM based VLSI architecture for Huffman coding with real-time optimized code word table," 
		Proceedings of IEEE International MidWest Symposium on Circuits And Systems (MWSCAS), 
		pp.~55--58, Aug., 2005.
\item
	\underline{Takeshi Kumaki}, Yutaka Kono, Masakatsu Ishizaki, Tetsushi Koide and Hans J\"urgen Mattausch, 
		``Application of multi-ported CAM for parallel coding," 
		Proceedings of IEEE Asia pacific Conference on Circuits and Systems (APCCAS), 
		pp. 1861-1864, Aug., 2006.

\end{itemize}
\noindent
{\bf 【国内研究会等発表】}
\begin{itemize}
\item
	\underline{熊木武志}, 岩井啓輔, 黒川恭一, 
	``多機能マルチポートCAMとその応用例, " 
	情報処理学会第64回全国大会, 
	Volume 1, pp.~55--56, Mar., 2001.
\item
	\underline{熊木武志}, 岩井啓輔, 黒川恭一, 
	``改良型多機能マルチポートCAMの提案, " 
	Forum on Information Technology 2002 (FIT2002), 
	Volume 1, No.~C-9, pp.~205--206, Sept., 2002.
\item
	鈴木諭司, 西山 怜, \underline{熊木武志}, 梶崎浩嗣, 岩井啓輔, 黒川恭一, 
	``CAMを用いた不正検知型侵入検知システム実装手法の提案, " 
	情報処理学会第65回全国大会, 
	Volume 3, pp.~555--556, Mar., 2002.
\item
	\underline{Takeshi Kumaki}, Yutaka Kono, Masakatsu Ishizaki, Tetsushi Koide and Hans J\"urgen Mattausch, 
		``CAM-based Huffman coding architecture for real-time applications and code-word-table optimizer," 
		Hiroshima International Symposium on Nanoelectronics for Tera-Bit Information Processing, 
		June, 2006.
\end{itemize}
\noindent
{\bf 【特許】}
\begin{itemize}
\item
	小出哲士, \underline{熊木武志}, 黒田泰斗, マタウシュハンスユルゲン, 
	野田英行, 堂阪勝己, 有本和民, 齋藤和則, 
	``符号化装置," 
	日本特許,  特願2005-146-211, 2005年5月19日出願.
\item
	行天隆幸, 堂阪勝己, 野田英行, 有本和民, 黒田泰斗, 
	\underline{熊木武志}, 小出哲士, マタウシュハンスユルゲン, 石崎雅勝, 
	``プロセッサおよびそれを用いたテーブル変換方法," 
	日本特許, 特願2006-283803, 2006年10月18日出願.
\end{itemize}
\noindent
{\bf 【その他研究活動】}		%ここに展示会参加等を書きましょう
\begin{itemize}
	\item
	小出哲士, \underline{熊木武志}, 黒田泰斗, マタウシュハンスユルゲン, 
	野田英行, 堂阪勝己, 有本和民, 齋藤和則, 
	``符号化装置," 
	日本特許,  特願2005-146-211, 2005年5月19日出願.
	\item
	行天隆幸, 堂阪勝己, 野田英行, 有本和民, 黒田泰斗, 
	\underline{熊木武志}, 小出哲士, マタウシュハンスユルゲン, 石崎雅勝, 
	``プロセッサおよびそれを用いたテーブル変換方法," 
	日本特許, 特願2006-283803, 2006年10月18日出願.
\end{itemize}
\clearpage

\end{document}