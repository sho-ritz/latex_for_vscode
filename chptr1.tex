%/**********************************************************************/
%		第1章:序論
%/**********************************************************************/
\pagenumbering{arabic}
\chapter{序論}
\label{lbl_chptr1}


%======================================================================
% 1.1 研究背景
%======================================================================
\section{研究背景}
近年,IoT(Internet of Things)技術の普及やスマートシティの実現に向けた取り組みが活発になる中で,
データ通信技術への需要はますます高まっている.
特に,位置情報に基づいた情報配信や,公共空間における情報提供システムの重要性が増している.
従来の無線通信技術(Wi-Fi,Bluetooth,携帯電話網など)は広く普及しているものの,
混雑した環境における通信品質の低下や,電磁波干渉の問題,更には通信インフラの設置コストといった課題が存在する.

可視光通信(Visible Light Communication: VLC)は,人の目に見える可視光線帯域の電磁波を用いて無線通信を行う技術である\cite{otsuka_vlc}.
LED等の光源が発する光が届く範囲で通信が可能となるため,一般に使われているLED照明を通信手段として流用できる点が特徴である.
また,可視光は生体への影響が小さく安全であり,電磁波による他機器への影響も少ないとされる.

一方,街中には大型のLEDマトリクスパネルを用いた広告看板や情報表示装置が数多く設置されている.
これらのパネルは,高解像度で鮮明な画像や動画を表示できるだけでなく,
高速な点滅制御が可能であることから,可視光通信の送信端末としての利用が期待されている.
特に,スマートフォンのカメラ機能を利用して受信を行う方式は,
ユーザーが専用の受信装置を用意する必要がなく,既存のデバイスで情報を受信できる点で実用性が高い.
例えば,街中の大型広告パネルにスマートフォンのカメラを向けるだけで,
その場所に関連する情報(店舗情報,イベント情報,クーポンなど)を自動的に取得できる可能性がある.

しかしながら,可視光通信を実用化する上では,いくつかの技術的課題がある.
第一に,人間の視覚に違和感を与えない範囲で情報を埋め込む必要がある.
高速な点滅は通信速度の向上に寄与するが,ちらつきとして知覚されると,
表示装置としての機能を損なう可能性がある.
第二に,スマートフォンカメラのフレームレートは通常30〜60 fps程度であり,
それよりも高い周波数で点滅させる場合,受信側での工夫が必要となってくる.
第三に,通信速度を向上させるためには,単一の領域で送信するよりも
複数の領域を並列に利用する必要があるが,
その際に人間の視覚に違和感を与えない範囲で情報を埋め込む必要がある.

これらのような技術的課題を解決するためには,LEDマトリクスパネルを可視光通信の送信端末として活用する上で,
高品質な画像表示の実現,遠隔からの表示制御,
視覚に違和感のない情報埋め込み方式の設計など,段階的に整えるべき技術要素がある.
本研究では,これらの技術的要素を順に構築し,
LEDマトリクスパネルを用いた可視光通信システムにおける送信部の構築を目指す.
\label{lbl_cp1_haikei}




%======================================================================
% 1.2 本研究の目的
%======================================================================
\section{本研究の目的}
\label{lbl_cp1_mokuteki}

本研究の目的は,LEDマトリクスパネルを用いた可視光通信システムにおける送信部を構築することである.
本研究は,段階的なアプローチにより,基礎的な表示技術の構築から始まり,
最終的な可視光通信システムの送信部実現に向けて実装を行った.
具体的には,以下の4段階で目標を設定した.

第一段階として,本研究で採用した点灯方式により256階調で表示するシステムを構築する.
これにより,ユーザーが指定した任意の画像をLEDマトリクスパネルに表示することができ,
実用的な画像表示システムの基盤を確立する.

第二段階として,クラウドシステムのCloudflare TunnelとWebアプリケーションのDjangoというWebアプリケーションフレームワークを使用して,
遠隔地から画像を指定し,その指定された画像を表示可能なサイズや形式にリサイズして
表示するシステムを確立する.
これにより,実際の運用環境を想定したシステム構成を実現し,
可視光通信システムの基盤となるインフラストラクチャを整備する.

第三段階として,人間の視覚に違和感を与えない明度に調整するためにガンマ補正の理論を活用し,
256階調表示システムを改良して補正を適用する.
これにより,高品質な画像表示を可能にし,
可視光通信システムにおいても視覚的に違和感のない表示を実現する基盤を構築する.

第四段階として,LEDマトリクスパネルを4つの領域に分割し,
それぞれの領域で異なる点滅周波数を用いてデータを並列に送信する方式を採用する.
これにより,単一領域で送信する方式と比較して4倍のデータ送信速度を実現することを目指す.

%======================================================================
% 1.3 本論文の構成
%======================================================================
\section{本論文の構成}
\label{lbl_cp1_kousei}

本論文は以下のように構成されている.

第2章では,可視光通信及びLED表示に関連する既存の研究を調査し,その内容をまとめる.

第3章では,LEDマトリクスパネルを用いた表示システムの構築について述べる.
まず,使用したHUB75規格の 64×32 LEDマトリクスパネルのハードウェア仕様と,
小型コンピュータのRaspberry Piを中心とした開発環境について説明する.
次に,本研究で採用した方式による256階調表示システムの実装方法を述べ,
画像のリサイズとマッピング処理,Webサイトからの画像アップロードから
パネル表示までの処理フローを説明する.
最後に,構築した表示システムの画質改善手法(ガンマ補正など)について評述する.

第4章では,LEDマトリクスパネルによる異なる点滅周波数を用いた可視光通信方式について述べる.
まず,点灯パターンと,4つの時間区間で輝度が等しくなる2種類のパターン(1010/1100)によるシンボル設計,4分割領域を用いた並列送信の構成を説明する.
次に,同期信号を用いない前提でのフレーム構成とフラグ検出によるシンボル境界推定の考え方を整理する.
更に,スマートフォンカメラのローリングシャッターによって生じる縞パターンを利用した
受信アルゴリズムの原理を述べる.

第5章では,提案方式の考察,及び今後の課題をまとめる.
第3章,及び第4章で述べた構成と方式の特徴を踏まえ,
実用化に向けた課題と今後の展望について議論する.
