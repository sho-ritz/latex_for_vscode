%/**********************************************************************/
%		第1章:序論
%/**********************************************************************/
\pagenumbering{arabic}
\chapter{序論}
\label{lbl_chptr1}


%======================================================================
% 1.1 研究背景
%======================================================================
\section{研究背景}
近年、IoT(Internet of Things)の普及やスマートシティの実現に向けた取り組みが活発化する中で、
データ通信技術への需要はますます高まっている.
特に、位置情報に基づいた情報配信や、公共空間における情報提供システムの重要性が増している.
従来の無線通信技術(Wi-Fi、Bluetooth、携帯電話網など)は広く普及しているものの、
混雑環境での通信品質の低下や、電磁波干渉の問題、さらには通信インフラの設置コストといった課題が存在する.

可視光通信(Visible Light Communication: VLC)は,人の目に見える可視光線帯域の電磁波を用いて無線通信を行う技術である.\footnote{株式会社大塚商会,「LED通信(可視光通信)とは」\texttt{https://www.otsuka-shokai.co.jp/}\allowbreak\texttt{products/led/knowledge/vlc.html}(最終閲覧日: 2026-01-18)}
LEDが発する光が届く範囲で通信が可能となるため,一般に使われているLED照明を通信手段として流用できる点が特徴である.
また,可視光は生体への影響が小さく安全であり,電磁波による他機器への悪影響も少ないとされる.

一方、街中には大型のLEDマトリクスパネルを用いた広告看板や情報表示装置が数多く設置されている.
これらのパネルは、高解像度で鮮明な画像や動画を表示できるだけでなく、
高速な点滅制御が可能であることから、可視光通信の送信端末としての利用が期待されている.
特に、スマートフォンのカメラ機能を利用して受信を行う方式は、
ユーザーが専用の受信装置を用意する必要がなく、既存のデバイスで情報を受信できる点で実用性が高い.
例えば、街中の大型広告パネルにスマートフォンのカメラを向けるだけで、
その場所に関連する情報(店舗情報、イベント情報、クーポンなど)を自動的に取得できる可能性がある.

しかし、可視光通信を実用化する上では、いくつかの技術的課題がある.
第一に、人間の視覚に違和感を与えない範囲で情報を埋め込む必要がある.
高速な点滅は通信速度の向上に寄与するが、ちらつきとして知覚されると、
表示装置としての機能を損なう可能性がある.
第二に、スマートフォンカメラのフレームレートは通常30〜60fps程度であり、
それよりも高い周波数で点滅させる場合、受信側での工夫が必要となってくる.
第三に、通信速度を向上させるためには、単一の領域で送信するよりも
複数の領域を並列に利用する必要があるが、
その際に人間の視覚に違和感を与えない範囲で情報を埋め込む必要がある.

本研究では、これらの課題を解決するため、LEDマトリクスパネルを用いた可視光通信システムの実現を最終目標として、
段階的なアプローチで研究を進めた.
まず、LEDマトリクスパネルの表示ロジックを理解するため、
RGB各色の点灯回数制御による256階調表示システムの実装から着手した.
この実装を通じて、HUB75インタフェースの制御方法やフレーム更新の仕組みを理解し、
高品質な画像表示を実現するための技術を習得した.

次に、実用化に向けた機能として、Djangoフレームワークを用いたWebアプリケーションを構築し、
Cloudflare Tunnelを利用して遠隔地から画像をアップロードしてLEDマトリクスパネルに表示するシステムを実現した.
この実装により、実際の運用環境を想定したシステム構成を確立し、
可視光通信システムの基盤となるインフラストラクチャを整備した.

さらに、表示品質の向上を目指し、ガンマ補正の理論を理解し実装した.
通常のディスプレイは自動的にガンマ2.2変換を行うため、
画像ファイルは予めガンマ0.45変換された状態で保存されている.
しかし、LEDマトリクスパネルではこの変換が行われないため、
元の画像よりも淡く表示される問題が発生していた.
ガンマ補正を実装することで、人間の視覚特性に合わせた適切な輝度表現を実現し、
高品質な画像表示を可能にした.

これらの基礎的な実装を通じて、LEDマトリクスパネルの制御技術を習得した後、
最終的な目標である可視光通信システムの実現に取り組んだ.
具体的には、1枚のLEDマトリクスパネルを4つの領域に分割し,
各領域で同一データを並列に送信する方式を採用した.
点灯パターンは4スロットで構成し,1010と1100のバランス符号により0/1を表現する.
両パターンともデューティ比が50\%となるため,4スロット平均の輝度が一致し,
人間の目には色味差やちらつきが生じにくい.
一方で,受信側は平均輝度ではなくローリングシャッター由来の縞パターンを用いて
点滅を検出する必要がある.
この構成により,単一領域で送信する場合と比較して4倍の並列送信が可能となる.

本技術の実用化により、将来的には街中の大型広告パネルや情報表示装置が
可視光通信の送信端末として機能し、通行人がスマートフォンのカメラを向けるだけで
位置情報に基づいた情報を自動的に取得できる社会の実現が期待される.
\label{lbl_cp1_haikei}




%======================================================================
% 1.2 本研究の目的
%======================================================================
\section{本研究の目的}
\label{lbl_cp1_mokuteki}

本研究の目的は、LEDマトリクスパネルを用いた可視光通信の送信部システムを確立することである.
本研究は、段階的なアプローチにより、基礎的な表示技術の習得から始まり、
最終的な可視光通信システムの送信部の実現に向けて実用化を目指した.
具体的には、以下の4つの段階的な目標を設定した.

第一に、RGB各色の点灯回数制御により256階調で表示するシステムを構築する.
これにより、ユーザーが指定した任意の画像をLEDマトリクスパネルに表示することができ、
実用的な画像表示システムの基盤を確立する.

第二に、Cloudflare TunnelとDjangoというWebアプリケーションフレームワークを使用して、
遠隔地から画像を指定し、その指定された画像を表示可能なサイズや形式にリサイズして
表示するシステムを確立する.
これにより、実際の運用環境を想定したシステム構成を実現し、
可視光通信システムの基盤となるインフラストラクチャを整備する.

第三に、人間の視覚に違和感を与えない明度に調整するためにガンマ補正の理論を理解し、
独自の256階調表示システムを改良して補正を適用する.
これにより、高品質な画像表示を可能にし、
可視光通信システムにおいても視覚的に違和感のない表示を実現する基盤を構築する.

第四に、LEDマトリクスパネルを4つの領域に分割し、
それぞれの領域で異なる点滅周波数を用いて同一のデータを並列に送信する方式を採用する.
これにより、単一領域で送信する方式と比較して4倍のデータ送信速度を実現することを目指す.

%======================================================================
% 1.3 本論文の構成
%======================================================================
\section{本論文の構成}
\label{lbl_cp1_kousei}

本論文は以下のように構成されている.

第2章では、LEDマトリクスパネルを用いた表示システムの構築について述べる.
まず、使用したHUB75 64×32 LEDマトリクスのハードウェア仕様と、
Raspberry Piを中心とした開発環境について説明する.
次に、RGB各色の点灯回数制御による256階調表示システムの実装方法を述べ、
画像のリサイズとマッピング処理、Webサイトからの画像アップロードから
パネル表示までの処理フローを説明する.
最後に、構築した表示システムの品質評価を行い、
ガンマ補正などの画質改善手法について議論する.

第3章では、LEDマトリクスパネルによる異なる点滅周波数を用いた可視光通信方式について述べる.
まず、点灯パターンとバランス符号によるシンボル設計、4分割領域を用いた並列送信の構成を説明する.
次に、同期語を用いない前提でのフレーム構成とFLAG検出によるシンボル境界推定の考え方を整理する.
さらに、スマートフォンカメラのローリングシャッターによって生じる縞パターンを利用した
受信アルゴリズムの方針を述べる.

第4章では、提案方式の評価と考察を行う.
表示品質と通信性能の評価結果を整理し,
実用化に向けた課題と今後の展望について議論する.

第5章は省略し,本研究のまとめと今後の課題は第4章で述べる.
