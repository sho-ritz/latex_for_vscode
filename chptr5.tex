%%%%%%%%%%%%%%%%%%%%%%%%%%%%%%%%%%%%%%%%%%%%%%%%%%%%%%%%%%%%%%
%%     第5章:評価および考察
%%%%%%%%%%%%%%%%%%%%%%%%%%%%%%%%%%%%%%%%%%%%%%%%%%%%%%%%%%%%%%
\chapter{評価および考察}
\label{lbl_chptr5_rslt}

本章では,第\ref{lbl_chptr3}章から第\ref{lbl_chptr5}章までで構築してきた
LEDマトリクス表示システムおよびフリッカ周波数差を用いた可視光通信方式について,
表示品質と通信性能の観点から評価を行い,その結果を踏まえて考察を行う.
まず実験条件を整理し,人間の視覚における見え方と通信性能(ビット誤り率など)を評価した後,
設計改善の流れと今後の課題についてまとめる.


%======================================================================
% 5.1 実験条件
%======================================================================
\section{実験条件}

本節では,表示品質評価および通信性能評価に共通して用いた実験条件を整理する.
使用したLEDパネルやRaspberry Piの型番,カメラ機種,送信距離,
撮影条件(露光時間,ISO感度,絞り値),室内照明条件などを簡潔に列挙し,
再現性のある実験環境であることを示す.


%======================================================================
% 5.2 人間の視覚における見え方の評価
%======================================================================
\section{人間の視覚における見え方の評価}

本節では,旧方式(デューティ比に差がある単純なフリッカ方式)と,
提案方式(平均デューティ50\%のバランス符号)を比較し,人間の目で見たときの色味と
違和感を評価する.

\subsection{旧方式と提案方式の色味の比較}

本小節では,両方式でパネルを駆動した際の見え方を写真や主観評価に基づいて比較する.
白色に対してどの程度黄ばみが感じられるか,ちらつき感があるか,といった観点から,
提案方式が色味差をどの程度低減できているかを定性的に整理する.

\subsection{白色との色差評価}

本小節では,簡易的な方法で白色との色差を評価する.
例えば,RGB値の平均や画像処理ソフトを用いた色差指標を参考にしつつ,
主観評価と合わせて,人間が許容できるレベルの色味差になっているかどうかを議論する.


%======================================================================
% 5.3 通信性能評価
%======================================================================
\section{通信性能評価}

本節では,提案方式を用いて既知bit列を送信し,受信側で復号したbit列と比較することで
通信性能を評価する.

\subsection{ビット誤り率と同期成功率の評価}

本小節では,距離・撮影角度・露光時間などの条件を変化させながら,
ビット誤り率(BER)や同期成功率を測定した結果について述べる.
どの条件で誤りが増加しやすいか,また同期語が誤検出されやすい条件は何かを整理する.

\subsection{送信可能な情報量と必要時間の試算}

本小節では,理論スループットと実際のBERを踏まえて,
実用的に送信可能なbit数(例:短いURLや簡単なメッセージ)と必要時間を概算する.
これにより,本方式がどのような用途(設定情報の配布,簡易なペアリング情報の送信など)に
適しているかを検討する.


%======================================================================
% 5.4 考察
%======================================================================
\section{考察}
\label{lbl_chptr5_ftr}

本節では,ステゴパネルからフリッカ周波数ベースの可視光通信方式へと至るまでの
設計改善の流れを振り返り,表示品質と通信性能のトレードオフについて考察する.
シンボル長や輝度差,同期語長をどのように設定すると,
人間の視覚的な違和感と通信の信頼性のバランスが取れるかを議論する.
また,誤り訂正符号や色チャネルの活用など,実用化に向けて検討すべき拡張点を列挙する.


%======================================================================
% 5.5 結論および今後の課題
%======================================================================
\section{結論および今後の課題}

本節では,本研究で得られた知見を簡潔にまとめる.
第\ref{lbl_chptr3}章で構築したマトリクスLED表示システム,
第\ref{lbl_chptr4}章で示したステゴパネル,第\ref{lbl_chptr5}章で提案した
フリッカ周波数差を用いた可視光通信方式が,順に発展しながら一つの枠組みとして
まとまっていることを整理する.
そのうえで,デジタルサイネージへの埋め込み情報配信や屋内位置情報タグ,
簡易ペアリング手段など,将来的な応用可能性と,さらなる高速化や安定化に向けた
ハードウェア/アルゴリズム側の改善課題を示し,本論文の結論とする.

