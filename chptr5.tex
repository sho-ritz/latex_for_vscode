%%%%%%%%%%%%%%%%%%%%%%%%%%%%%%%%%%%%%%%%%%%%%%%%%%%%%%%%%%%%%%
%%     第5章:考察および今後の課題
%%%%%%%%%%%%%%%%%%%%%%%%%%%%%%%%%%%%%%%%%%%%%%%%%%%%%%%%%%%%%%
\chapter{考察及び今後の課題}
\label{lbl_chptr5_rslt}

本章では,第\ref{lbl_chptr3}章,及び第\ref{lbl_chptr4}章で構築してきた
LEDマトリクス表示システムおよび2つの異なる点滅周波数を用いた可視光通信方式について,
構成と方式の特徴を踏まえて考察し,今後の課題についてまとめる.

% %======================================================================
% % 5.2 人間の視覚における見え方の評価
% %======================================================================
% \section{人間の視覚における見え方の評価}

% 本節では,非対称デューティ比方式(デューティ比に差がある方式)と本方式(平均デューティ50\%のバランス符号)を比較し,人間の目で見たときの色味と違和感を定量的に評価する.

% \subsection{非対称デューティ比方式と本方式の色味の比較}
% \ref{lbl_chptr3}章で述べた非対称デューティ比方式と本方式を比較する.非対称デューティ比方式では,4スロットの範囲での光量がビット1とビット0で異なる(点灯数が4回と2回)ため,高い点滅周波数の領域では光量が多く,低い点滅周波数の領域では光量が少なくなる.その結果,明暗の差が大きく,視覚的にちらつきが目立つ.一方,本方式では両方のパターンとも4スロット中で点灯が2回でありデューティ比が50\%で揃うため,4スロット平均の輝度が等しくなり,色味差や明暗の差が生じにくい.

% \subsection{駆動周波数ごとのちらつきの定量的評価}
% 駆動周波数(120Hz,240Hz,480Hz)ごとに,ちらつきの主観評価を定量的に実施した.評価方法として,5段階の主観評価尺度を用いた.1を「ちらつきが非常に気になる」,5を「ちらつきを全く感じない」と定義し,各駆動条件で同一の表示パターン(バランス符号による点滅)を複数回提示し,評価者が毎回1--5のスコアを付けた.各条件で得られたスコアの平均値と標準偏差を算出し,駆動周波数との関係を比較した.

% 評価結果を表\ref{tab:flicker_eval}に示す.120Hz駆動(点滅周波数60Hz/30Hz)では平均スコアが2.2(標準偏差0.5)と低く,30Hz成分が視覚に残りやすくちらつきが目立つことが数値として表れた.240Hz駆動(点滅周波数120Hz/60Hz)では平均3.0(標準偏差0.4)となり,120Hz駆動より改善したが,まだちらつきを感じる評価が多かった.480Hz駆動(点滅周波数240Hz/120Hz)では平均4.2(標準偏差0.3)となり,臨界フリッカ融合周波数(CFF)を上回る成分が増えたことで,ほとんどちらつきを感じないと評価されることが定量的に確認された.以上より,駆動周波数を高くするほど主観スコアが上昇し,ちらつきが抑制されることが定量的に示された.

% \begin{table}[htbp]
%   \centering
%   \caption{駆動周波数ごとのちらつき主観評価(5段階尺度の平均値と標準偏差)}
%   \label{tab:flicker_eval}
%   \begin{tabular}{lccc}
%     \hline
%     駆動周波数 & 点滅周波数(高/低) & 平均スコア & 標準偏差 \\
%     \hline
%     120Hz & 60Hz / 30Hz & 2.2 & 0.5 \\
%     240Hz & 120Hz / 60Hz & 3.0 & 0.4 \\
%     480Hz & 240Hz / 120Hz & 4.2 & 0.3 \\
%     \hline
%   \end{tabular}
% \end{table}

%======================================================================
% 5.4 考察
%======================================================================
\section{考察}
\label{lbl_chptr5_ftr}

今回LEDマトリクスパネルを用いて256階調での画像の表示,外部から画像をアップロードして表示を行えるシステムの構築,
LEDマトリクスパネルを用いた可視光通信の実験を行った.
第\ref{lbl_chptr3}章,及び第\ref{lbl_chptr4}章で述べた構成と方式の特徴から,表示品質と通信性能の間には明確なトレードオフがあることが分かる.非対称デューティ比方式ではデューティ比の差により
周波数ごとの明るさが不均一となり,ちらつきと色味差が目立った.一方,対称デューティ比方式では4スロット平均で輝度が一致するように設計したため,人間の目には白色の揺らぎが少なくなり,表示装置としての違和感が大きく改善された.

周波数条件については,120 Hz/240 Hzでは低周波成分が残り,ちらつきが視認できたのに対し,480 Hzではほぼ連続点灯に見えた.
ただし,高周波化は縞パターンのピッチを細かくし,撮影条件によっては縞のコントラストが低下する可能性がある.
すなわち,視覚的な快適さを優先すると通信の検出が難しくなりやすく,逆に検出性を優先するとちらつきが増えるという
関係がある.

また,本方式は平均輝度が一致するため,領域内の平均輝度からはビット差が得られず,
受信側はローリングシャッターによって生じる縞パターンを用いて復号する必要がある.
この設計により視覚的品質は高いが,受信アルゴリズム側の前処理や位相推定に依存する部分が大きく,
撮影距離や角度,露光条件の変化に対しても安定して復号ができるようにするという点が今後の課題である.

更に,4領域並列化は通信速度を向上させる一方,1領域あたりの画素数が減少するため,
縞パターンの判別が難しくなる場合がある.実運用を想定する場合は,距離や表示サイズに応じて
分割数やシンボル長を調整する必要があると考えられる.




%======================================================================
% 5.5 結論および今後の課題
%======================================================================
\section{結論及び今後の課題}
本研究では,LEDマトリクスパネルを用いた表示システムの構築と画質改善を行い,
その上で2つの異なる点滅周波数を用いた可視光通信方式を提案した.
表示面では,256階調表示とガンマ補正により視覚的な違和感を抑え,
通信面では4分割領域と本研究で提案した点灯方式を用いることで,ちらつきを抑えつつ情報を埋め込めることを示した.
一方で,受信側の実装と実測による通信性能評価は十分に行えておらず,
縞パターン抽出や位相推定の頑健性を含めた検証が今後の課題である.

今後は,表示・通信・受信アルゴリズムを統合したシステムとして実装し,
撮影距離や角度,露光条件に対する復号性能の評価を進める必要がある.
また,誤り訂正符号やシンボル長の最適化,分割数の設計指針の整備を通じて,
更なる実用性や安定性の向上を目指したい.
