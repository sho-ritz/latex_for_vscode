%/**********************************************************************/
%		第4章:LEDマトリクスのフリッカ周波数差を用いたカメラベース可視光通信方式
%**********************************************************************/
\chapter{2つの異なる点滅周波数を用いた可視光通信方式}
\label{lbl_chptr4}

第\ref{lbl_chptr3}章では,LEDマトリクスパネルによる256階調表示,ガンマ補正,及び遠隔からの画像アップロードと表示を行うシステムの構築について述べた.本章で扱う可視光通信の方式(点灯パターンや分割領域の設計)は,表示システムの実装に直接依存するものではない.一方で,実用化後には同一のパネルで,任意の画像を表示しつつその上に情報を埋め込んだ可視光通信を行うことが想定される.従って,第\ref{lbl_chptr3}章で示した「表示として見やすい」基盤の上に,本章の通信方式を組合わせることで,表示と通信の両立を目指す構成となっている.以下では,2つの異なる点滅周波数を用いた可視光通信方式について評述する.

%======================================================================
% 4.1 提案方式の概要
%======================================================================
\section{提案方式の概要}
本章では,LEDマトリクスパネルを用いた可視光通信方式について説明する.
提案する方式は,2つの異なる点滅周波数を用いてバイナリを表現してデータを送信する.また,一枚のLEDマトリクスパネルを複数の領域に分割し,一回で送信できるデータ量の増加を目指した.これらについて説明する.


%======================================================================
% 4.2 送信側ハードウェア・ソフトウェア構成
%======================================================================
\section{可視光通信の仕組み}

本節では,具体的な可視光通信の仕組みについて説明する.具体的には点灯パターンと分割領域の設計について説明する.


\subsection{点灯パターンの設計}
提案するシステムでは,2つの点滅周波数を用いて,バイナリを表現し,それを受信部で復号する方式を採用した.
2つの点滅周波数を用いたバイナリの表現方法について説明する.

まず用語を定義する.スロットとは,時間軸を等間隔に区切った一区間であり,点灯と消灯とを切り替える最小の時間単位である.送信側では,このスロットごとにLEDを点灯するか消灯するかを決めてパターンを生成する.シンボルとは,1ビット(0または1)を表現するために複数のスロットをまとめた単位である.本方式では1シンボルを4スロットで構成する.以上を踏まえ,点灯パターンの具体を示す.

本方式では,1シンボルを4スロットで構成し,スロットごとの点灯(1),及び消灯(0)の並びで2種類のパターンを用いる.前者のパターンは1010(1スロットごとに点灯と消灯が交互に現れる)とし,後者のパターンは1100(2スロット点灯ののち2スロット消灯)とする.1010は4スロットの間に点滅が2回生じるため相対的に高い点滅周波数となり,1100は点滅が1回となるため相対的に低い点滅周波数となる.受信側では,観測した4スロットのパターンが1010に近ければビット1,1100に近ければビット0として復号する.このように,2つの異なる点滅周波数(高・低)をビット1・0に対応させることでバイナリを表現する.

当初は,領域を4つに分割することを中心に検討していたため,全体を高い方の周波数で駆動させ,低い方の周波数領域は1/2になるように,2回に1回点灯させる方式を検討していた.しかしながら,この方法では,デューティ比が高い周波数の時は100\%,低い方の周波数の時は50\%となり,両者で非対称となる.これは,単位時間あたりで人間の目に入る光量が高周波領域の方が多くなってしまうため,人間の視覚的には激しいちらつきが発生してしまう.そのため,この方式は採用しなかった.(以下,この方式を「非対称デューティ比方式」と呼ぶ)
それを考慮して新たに点灯パターンを設計した.具体的には,両方ともデューティ比が50\%となるようにする方式である(以下,この方式を「対称デューティ比方式」と呼ぶ).

非対称デューティ比方式と対称デューティ比方式を比較したものを図\ref{fig:flicker_pattern}に示す.

\begin{figure}[H]
  \centering
  \includegraphics[width=0.8\linewidth]{assets/flicker_pattern.png}
  \caption{提案方式におけるスロットパターン}
  \label{fig:flicker_pattern}
\end{figure}

図\ref{fig:flicker_pattern}では,赤色の領域がビット1を表すスロットパターン,青色の領域がビット0を表すスロットパターンを示している.

非対称デューティ比方式(図上側)では,ビット1が1111(4スロットすべて点灯),ビット0が1010(2スロット点灯)である.4スロット中の点灯数がビット1では4回,ビット0では2回と異なるため,デューティ比が100\%と50\%で非対称になる.一方,対称デューティ比方式(図下側)では,ビット1が1010,ビット0が1100であり,いずれも4スロット中で点灯が2回である.従って両方ともデューティ比が50\%となり,対称である.

本方式の2つのスロットパターン(1010と1100)は,4スロットで見たときの輝度が等しくなるように設計されている.以降,これらを「バランス符号」と呼ぶ.

このバランス符号を用いることで,ビット1とビット0のいずれもデューティ比50\%となり,4スロットで見たときの輝度が揃うため,人間の視覚に違和感が生じにくい.

\subsection{バイナリ変換の方法}
受信側では,\ref{sec:256gradation}節で述べたバランス符号に従い,観測した4スロットのパターンが1010に近ければビット1,1100に近ければビット0として復号する.この処理を時間方向に繰り返し適用することで,点灯パターンからビット列を得る.例として,1001というビット列を表現する場合のスロットパターンを図\ref{fig:1001_pattern}に示す.
\begin{figure}[H]
  \centering
  \includegraphics[width=0.8\linewidth]{assets/1001_bit.png}
  \caption{1001というビット列を表現する場合のスロットパターン}
  \label{fig:1001_pattern}
\end{figure}

これは,ビット列「1001」をバランス符号で表現した例である.左から順にビット1,0,0,1に対応し,それぞれのシンボルが1010,1100,1100,1010のスロットパターンになっている.送信側では,このようにビット列を4スロットごとのバランス符号に置き換え,時間方向に並べることで点灯パターンを生成する.

動作周波数については,当初は全体を120 Hz(1スロット$=$1/120 s)で駆動した.このとき,ビット1のパターン1010は点灯と消灯が2スロットごとに繰り返すため,点滅の1周期は2スロット($=$1/60 s)であり,実質的な点滅周波数は60 Hzとなる.一方,ビット0のパターン1100は4スロットで1周期となるため,点滅周波数は30 Hzとなる.従って,120 Hz駆動ではビット1が60 Hz,ビット0が30 Hzという2つの点滅周波数で送信されていたこととなる.その結果,この周波数では,人間の目にちらつきが発生してしまう.

人間の視覚には臨界フリッカ融合周波数(CFF:critical flicker fusion)があり,この周波数以上ではフリッカを感じず連続点灯に見える.CFFは条件により異なるが,おおよそ60 Hzから100 Hzとされる.そこで,高い方の点滅周波数が120 Hz(CFFを上回り,連続点灯に見えやすい)となるように,全体を480 Hzで駆動するようにした.1スロットは1/480sであり,4スロットで1シンボルとなる.

\subsection{撮影画像における縞模様}
後述する図\ref{fig:split_matrix_4}のように,点滅するLEDマトリクスパネルをカメラで撮影すると,画面上に縞模様が現れることがある.これは,多くのスマートフォンカメラが採用するローリングシャッター方式によるものである.ローリングシャッターでは,1回の撮影の間にセンサの読出しが上から下へ順に行われるため,画面上の縦位置によって光を取り込む時刻が異なる.送信側ではパネルがスロットごとに点灯・消灯を繰り返しているため,読出しの時刻が点灯区間と重なった行は明るく,消灯区間と重なった行は暗く記録される.その結果,画像上では明暗が縞状に並んで見える.実際に縞模様が現れた画像を図\ref{fig:stripe_pattern}に示す.
\begin{figure}[H]
  \centering
  \includegraphics[width=0.8\linewidth]{assets/flicker_noise.jpg}
  \caption{縞模様が現れた画像}
  \label{fig:stripe_pattern}
\end{figure}

この縞模様は,1フレーム内に時間方向の点滅情報が縦方向に展開されたものと解釈できる.従って,縞のパターンを解析することで,カメラのフレームレート(例:60 fps)で区切られた時間分解能を超えて,ビット列を推定できる可能性がある.今回は受信側の復号ロジックまで十分に設計・検証できていないが,今後の課題として,縞パターンからのビット推定や,フレームレートを上回るデータ受信の実現が考えられる.

\subsection{分割領域における点灯パターンの設計}
ここでは,一枚のLEDマトリクスパネルを4つの領域に分割し,それぞれの領域で異なる点滅周波数を用いて同一のデータを並列に送信する方式を採用した.
\begin{figure}[H]
  \centering
  \includegraphics[width=0.8\linewidth]{assets/split_matrix.png}
  \caption{4分割LEDマトリクスの走査制御}
  \label{fig:split_matrix}
\end{figure}

図\ref{fig:split_matrix}のような場合,受信側では1001というデータに変換することができる.このように,分割領域における点灯パターンを設計することで,一回で送信できるデータ量を増加させることができる.
この方法で実際に点灯した画像を図\ref{fig:split_matrix_4}に示す.

\begin{figure}[H]
  \centering
  \includegraphics[width=0.8\linewidth]{assets/splite_matrix_4.jpg}
  \caption{LEDマトリクスパネルを4つの領域に分割した画像}
  \label{fig:split_matrix_4}
\end{figure}

これまでの\ref{sec:256gradation}節,及び\ref{sec:lighting_comparison}節では,送信側の符号化(バランス符号でビットを4スロットの点灯パターンに置き換える)と,受信側の復号の原理(観測した4スロットのパターンから1010ならビット1,1100ならビット0とする)を述べた.図\ref{fig:split_matrix_4}はカメラで撮影した1枚の写真となる.今後は受信側のシステムを構築していく.


\subsection{4分割以上の検証}
本節では,4分割を前提に設計しつつ,8分割についても試験的に検証した.

8分割で点灯した場合,図\ref{fig:splited_8}のようになる.
\begin{figure}[H]
  \centering
  \includegraphics[width=0.8\linewidth]{assets/splited_8.jpg}
  \caption{LEDマトリクスパネルを8つの領域に分割した画像}
  \label{fig:splited_8}
\end{figure}

このように,送信側には分割数を増やす余地がある.受信側で安定して読み取れるのであれば,分割数を更に増やすことで,より高速な通信が実現できると考えられる.
%======================================================================
% 4.4 フレーム構成とシンボル境界推定
%======================================================================
\section{フレーム構成と復号の設計}
本節では,送信フレームの構成,フラグ検出とシンボル境界推定,及び受信アルゴリズムの方針について述べる.

\subsection{送信フレームの構成}
送信するデータの全体設計は図\ref{fig:data_structure}の通りである.
\begin{figure}[H]
  \centering
  \includegraphics[width=0.8\linewidth]{assets/data_structure.png}
  \caption{送信データの全体設計}
  \label{fig:data_structure}
\end{figure}

フラグ(FLAG)はデータの開始位置と終了位置を示すものであり,Payloadは実際に送信するデータの内容,CRCはビット誤り検出を行うものである.
この設計はHDLCのフレーム同期方式を参考にしている\cite{lineeye_hdlc}.


\subsection{フラグ検出とシンボル境界推定}
今回の方式では,フラグを用いてデータの開始位置と終了位置を示す.今回,フラグは 01111110 としたが,ペイロード内に 01111110 が含まれていた場合,そのビット列をフラグとして誤検出してしまう問題がある.そこで本研究ではビットスタッフィングを適用し,ペイロード中に 1 が5回連続して現れた場合,その直後に 0 を挿入する.受信側では,1 が5回連続した後に現れる 0 を取り除く(デスタッフィング)ことで元のペイロードを復元する.この処理により,ペイロード中にFLAGと同一のビット列が出現しないことが保証され,フラグの検出が可能となる.

シンボル境界推定について,本研究ではバランス符号として 1100 を0,1010 を1に割り当て,1シンボルを4スロットで構成する.受信側でシンボル境界(位相)がずれた場合,4スロット窓で観測されるパターンは 1100 が 1001 や 0011 となるなど,送信時の並びと一致しなくなる.そのため,受信信号を1スロットずつシフトさせながら候補パターンとの整合を評価し,最も整合する位相をシンボル境界として採用する.特に 1010 は位相をずらしても 1010 または 0101 の交互パターンとなり,連続する 1 や 0 が生じにくい.この性質は位相推定における識別性を高める点で,本方式のシンボル設計の利点である.

ただし,カメラ撮像ノイズ等により,位相ずれではなく誤りによって 1001 のようなパターンが観測される可能性もある.そこで本研究では,ペイロード末尾にCRCを付与し,受信側で復号結果の整合性を検証する.具体的には,位相候補(0〜3スロットのシフト)ごとに復号を行い,CRC検証に合格する候補を正しい復号結果として採用する.これにより,シンボル境界推定の誤りや伝送中のビット誤りを高確率で検出でき,誤ったデータ列を上位層へ渡すことを防ぐ.


%======================================================================
% 4.5 受信アルゴリズム
%======================================================================
\subsection{受信アルゴリズム}

本節では,スマートフォンカメラで撮影されたフレーム列からビット列を復号するアルゴリズムについて述べる.

\subsubsection{フレーム列からの時系列データ生成}
受信側では,60 fps程度で取得した動画をフレームに分解し,各フレームに対して4分割領域内の縞パターンを抽出する.本方式は4スロットで平均輝度が一致するように設計しているため,領域内の平均輝度からはビット差が得られない.そのため,ローリングシャッターによって生じる縞のパターンからビットの推定を行う.得られた縞プロファイルを時間方向に並べることで,領域ごとの時系列データを生成する.

\subsubsection{ビットパターンとの類似度評価と位相推定}
送信側では1シンボルを4スロット(1スロット=1/480 s)で構成しているため,受信側の60 fpsでは1シンボルあたりの観測点が不足する可能性がある.そこで,縞パターンから得られた時系列データに対して,理論パターン1010,及び1100に対応する縞の並びと一致するかを評価する.

また,ローリングシャッターやフレームレートの揺らぎによる位相ずれに対応するため,開始位置を1フレームずつずらして複数の位相候補を評価する.各位相で得られたスコアが最大となるパターンを選択し,1010なら1,1100なら0として復号する.これにより,同一シンボル長でも撮影条件に依存しない復号が可能になる.

\subsection{復号フロー全体}
復号フローは,大きく「前処理」「フラグ検出」「ペイロード復号」の3段階で構成される.まず前処理としてフレーム列から縞パターン時系列を生成し,正規化を行う.次に,フラグパターンと一致する区間を探索してフレーム境界を決定する.このとき,位相をずらした複数候補を評価し,最も一致度の高い位置を採用する.

境界が確定した後は,ペイロード領域に対してシンボルごとの類似度評価を行い,ビット列を復号する.誤検出が発生した場合は,フラグ再検出に戻ることで再同期を行い,長いペイロードでも復号を継続できる構成とした.
