%%%%%%%%%%%%%%%%%%%%%%%%%%%%%%%%%%%%%%%%%%%%%%%%%%%%%%%%%%%%%%
%%     第5章:評価および考察
%%%%%%%%%%%%%%%%%%%%%%%%%%%%%%%%%%%%%%%%%%%%%%%%%%%%%%%%%%%%%%
\chapter{評価および考察}
\label{lbl_chptr5_rslt}

本章では,第\ref{lbl_chptr3}章から第\ref{lbl_chptr5}章までで構築してきた
LEDマトリクス表示システムおよびフリッカ周波数差を用いた可視光通信方式について,
表示品質と通信性能の観点から評価を行い,その結果を踏まえて考察を行う.
まず実験条件を整理し,人間の視覚における見え方等を評価し,
設計改善の流れと今後の課題についてまとめる.

%======================================================================
% 5.2 人間の視覚における見え方の評価
%======================================================================
\section{人間の視覚における見え方の評価}

本節では,旧方式(デューティ比に差がある単純なフリッカ方式)と,
提案方式(平均デューティ50\%のバランス符号)を比較し,人間の目で見たときの色味と
違和感を評価する.

\subsection{旧方式と提案方式の色味の比較}
\ref{lbl_chptr3}章で提案した旧法と新法を比較する。旧法では、4スロットの範囲での光量が違うため、高い周波数の方が光量が多くなり、低い周波数では光量が少なくなる。その影響で明暗が激しく切り替わりチカチカしていた。一方、新法では、両周波数ともデューティ比が50\%となるように設計しているため、人間の目には白色と黄色が混在したような色味差が生じない。また、新法では、4スロットの範囲での光量が揃うように設計しているため、人間の目には白色と黄色が混在したような色味差が生じない。

\subsection{周波数ごとの色差評価}
周波数ごとの色差についてだが、自分が検証したのは120Hz駆動と240Hz駆動と480Hz駆動の3つである。この3つの周波数について、色差を評価する。
第一に、120Hz駆動だが、120Hz駆動の場合は60Hzと30Hzの点滅周波数となるが、30Hz成分が際立ちやすい。そのため、人間の目からはかなりちらつきが目立った。
第二に、240Hz駆動だが、240Hz駆動の場合は120Hzと60Hzの点滅周波数となるため、先ほどよりもちらつきが少なくはなったがまだ若干ちらつきを感じた。
第三に、480Hz駆動だが、480Hz駆動の場合は240Hzと120Hzの点滅周波数となり、第一、第二と比較してほとんどちらつきを感じなくなった。

%======================================================================
% 5.4 考察
%======================================================================
\section{考察}
\label{lbl_chptr5_ftr}

今回LEDマトリクスパネルを用いて256階調での画像の表示、外部から画像をアップロードして表示を行えるシステムの構築、
LEDマトリクスパネルを用いた可視光通信の実験を行った。
本章の評価結果から,表示品質と通信性能の間には明確なトレードオフがあることが分かる。旧方式ではデューティ比の差により
周波数ごとの明るさが不均一となり,ちらつきと色味差が目立った。一方,提案方式では4スロット平均で輝度が一致するように
設計したため,人間の目には白色の揺らぎが少なくなり,表示装置としての違和感が大きく改善された。

周波数条件については,120Hz/240Hzでは低周波成分が残り,ちらつきが視認できたのに対し,480Hzではほぼ連続点灯に見えた。
ただし,高周波化は縞パターンのピッチを細かくし,撮影条件によっては縞のコントラストが低下する可能性がある。
すなわち,視覚的な快適さを優先すると通信の検出が難しくなりやすく,逆に検出性を優先するとちらつきが増えるという
関係がある。

また,提案方式は平均輝度が一致するため,領域内の平均輝度からはbit差が得られず,
受信側はローリングシャッターによって生じる縞パターンを用いて復号する必要がある。
この設計により視覚的品質は高いが,受信アルゴリズム側の前処理や位相推定に依存する部分が大きく,
撮影距離や角度,露光条件の変化に対しても安定して復号ができるようにするという点が今後の課題である。

さらに,4領域並列化は通信速度を向上させる一方,1領域あたりの画素数が減少するため,
縞パターンの判別が難しくなる場合がある。実運用を想定する場合は,距離や表示サイズに応じて
分割数やシンボル長を調整する必要があると考えられる。




%======================================================================
% 5.5 結論および今後の課題
%======================================================================
\section{結論および今後の課題}
本研究では,LEDマトリクスパネルを用いた表示システムの構築と画質改善を行い,
その上でフリッカ周波数差を用いた可視光通信方式を提案した.
表示面では,256階調表示とガンマ補正により視覚的な違和感を抑え,
通信面では4分割領域と自分が提案した点灯方式を用いることで,ちらつきを抑えつつ情報を埋め込めることを示した.
一方で,受信側の実装と実測による通信性能評価は十分に行えておらず,
縞パターン抽出や位相推定の頑健性を含めた検証が今後の課題である.

今後は,表示・通信・受信アルゴリズムを統合したシステムとして実装し,
撮影距離や角度,露光条件に対する復号性能の評価を進める必要がある.
また,誤り訂正符号やシンボル長の最適化,分割数の設計指針の整備を通じて,
更なる実用性や安定性の向上を目指したい.
