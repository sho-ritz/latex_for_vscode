%/**********************************************************************/
%		第2章:関連研究
%/**********************************************************************/
\chapter{関連研究}
\label{lbl_chptr2}

本章では,LEDマトリクスパネルを用いた可視光通信と画像表示に関連する既存の研究をまとめ,本研究の背景と位置づけを述べる.まず可視光通信の概要と応用,次にカメラで受信する方式,更に表示と通信を両立させるための「ちらつきを抑える」設計についての研究を紹介する.最後に,これらの関連研究と本研究との関係を述べる.

%======================================================================
% 2.1 可視光通信の概要と応用
%======================================================================
\section{可視光通信の概要と応用}

可視光通信(VLC)は,可視光を使って近距離でデータを送る通信方式である.LEDの明るさを変化させることで情報を載せ,光が届く範囲で通信できる\cite{ieee8021572018}.国際的な通信規格であるIEEE 802.15.7では,この方式の通信の仕様が定められており,照明として使うことを前提に,調光やちらつき抑制のための設計の考え方が整理されている\cite{rajagopal2012}.可視光通信は電波同士の干渉を受けにくいため,屋内での通信や位置の測定,情報の配信などへの応用が検討されている.商用化に向けた課題と展望も議論されている\cite{jovicic2013}.

従来の可視光通信では,光を電気信号に変える専用の受信装置が必要であった.一方,スマートフォンのカメラで受信する方式は,特別な装置がなくても既存の機器で情報を得られるため,実用化が期待されている.

%======================================================================
% 2.2 カメラを用いた可視光通信
%======================================================================
\section{カメラを用いた可視光通信}

カメラで受信する方式では,送信側でLEDの点灯・消灯のパターンを決め,受信側でカメラが撮った画像や動画からそのパターンを読み取り,データを取り出す.このとき,カメラがどのように光を取得するかが,受信のしやすさに大きく影響する.

多くのスマートフォンカメラは,画面を上から下へ順に一行ずつ光を読み取る方式を採用している.このため,画面上の縦の位置によって,光を取り込む「時刻」がずれる.点滅しているLEDを撮影すると,ある行は点灯しているときに写り,別の行は消灯しているときに写るため,画像には明暗の縞模様が現れる.つまり,点滅の「時間の流れ」が,画像の「縦方向の位置」に広がって表れる.この縞模様を解析すれば,1枚の写真からでも,短い時間ごとの点灯・消灯の並び(0と1の並び)を推定できる可能性がある.
例として図\ref{fig:rolling_shutter}の場合を考える.
\begin{figure}[htbp]
    \centering
    \includegraphics[width=0.7\textwidth]{assets/rolling_shutter_example.png}
    \caption{ローリングシャッター方式のカメラで撮影したLEDの点滅の例}
    \label{fig:rolling_shutter}
\end{figure}

図\ref{fig:rolling_shutter}は,マトリクスLEDパネルをローリングシャッター方式のカメラで撮影した例である.ローリングシャッターでは,LEDの光を上から下へ走査しながら各行を異なる時刻に取得するため,LEDの高速点滅が行方向に時間差として現れ,画像中に明暗の縞模様が生じる.この縞模様を解析することで,点滅パターンを推定し,送信データを復元できる可能性がある.

近年は,専用の受信装置が不要な「カメラで受信する方式」が注目されている.また,光を使った近距離通信の国際規格(IEEE 802.15.7)に基づく交通分野の可視光通信の実装例が報告されている\cite{belle2013its}.Teliらは,LEDを8×8個並べた送信機とカメラを使い,20--100 cmの距離で毎秒120--960ビット,最大1.8 mで毎秒13.44 kbpsの通信を実現したと報告している\cite{teli_rolling_mimo_2020}.また,カメラの撮影枚数(フレームレート)を超える速さでデータを送るため,縞模様からデータを取り出す方式も研究されている\cite{scsk_rolling_shutter}.本研究でも,この縞模様を利用して受信する方針とし,送信側ではLEDマトリクスパネルを複数の領域に分け,各領域で異なる点滅の速さを使って,同時にデータを送る構成を採用している.

%======================================================================
% 2.3 LED表示とちらつきの制御
%======================================================================
\section{LED表示とちらつきの制御}

可視光通信では,LEDの点灯・消灯を速く切り替えて情報を載せるが,「表示」としても使う場合には,人が見たときに「ちらつき」として気にならない範囲に収める必要がある.表示と通信を両立させるには,照明として自然に見える範囲で,点滅の速さや「点灯している時間の割合」を決め,目にわかるちらつきを抑える設計が求められる.IEEE 802.15.7では,照明として使うときの条件のもとで,データの送り方と明るさの調整・ちらつきの抑制についての考え方が定められており\cite{ieee8021572018}\cite{rajagopal2012},規格で定められた通信方式の設計の意図として,ちらつき・明るさの調整とデータの送り方の関係が整理されている\cite{roberts2011phy}.

人の目には,「これより速く点滅すると,ずっと点灯しているように見える」という臨界融合周波数がある\cite{davis2015flicker}.この値は条件によって異なるが,一般的に60–100 Hz程度とされる.\cite{davis2015flicker}.LEDの明るさ調整とデータ伝送を両立させる方式として,点灯時間の長さで表す方式(Pulse Width Modulation)や,点滅の回数で表す方式(Pulse Frequency Modulation)などが用いられる.本研究では,「ビット0とビット1で平均の明るさが違って見える」ことを避けるため,4つの時間区切りの間で点灯が2回(点灯している時間の割合が50\%)となるパターン(1010と1100)を採用している.このようにすることで,視覚的な違和感を抑えつつビットを送信している.

%======================================================================
% 2.4 本研究の位置づけ
%======================================================================
\section{本研究の位置づけ}

可視光通信では,カメラで受信する方式,カメラの「上から下へ順に写し取る」性質(ローリングシャッター方式)によってできる縞模様からデータを取り出す方法,及び表示と通信の両立のための「ちらつきを抑える」設計が,関連研究として進められている.

本研究では,LEDマトリクスパネルを送信側として使い,(1)256階調表示とガンマ補正による見やすい画像表示,(2)2種類の点滅の速さと,点灯割合が50\%になるパターン(1010/1100)による,ちらつきを抑えた0と1の送信,及び(3)パネルを4つの領域に分けて同時に送る構成,を組合わせた方式を提案している.関連研究で用いられている「縞模様の利用」や「点灯の割合・ちらつきの考慮」と同様の考え方を,LEDマトリクスパネルとその駆動方式(HUB75)を用いた具体的な実装としてまとめ,表示の質と通信の両立を目指している点に,本研究の特徴がある.
