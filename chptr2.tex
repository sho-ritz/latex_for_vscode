%/**********************************************************************/
%		第2章:マトリクスLED表示システムの構築
%/**********************************************************************/
\chapter{マトリクスLED表示システムの構築}
\label{lbl_chptr3}


%======================================================================
% 2.1 使用デバイスと開発環境
%======================================================================
\section{使用デバイスと開発環境}

本節では,使用したHUB75 64×32 LEDマトリクスのハードウェア仕様と,
Raspberry Piを中心とした開発環境についてまとめる.

\subsection{HUB75 64×32 LEDマトリクスの仕様}
% マトリクスの仕様について説明する。


\begin{figure}[H]
  \centering
  \includegraphics[width=0.8\linewidth]{assets/matrix_led.png}
  \caption{HUB75 64×32 LEDマトリクスの外観例}
  \label{fig:matrix_led}
\end{figure}

\subsection{Raspberry Pi,Webサーバ構成}

本小節では,Raspberry Pi上での開発環境構成について述べる.
LEDパネル制御コードとWebサーバ(画像アップロード用)の配置関係を示す.
さらに,LAN内のPCやスマートフォンからブラウザ経由で画像を送信し,
Raspberry Pi側で受信・変換してパネルに表示するまでの役割分担を簡潔に整理する.


%======================================================================
% 2.2 256階調表示システム
%======================================================================
\section{256階調表示システム}

\subsection{RGB各色の点灯回数制御による256階調表現}

本小節では,PWMもしくは時間多重方式を用いて1フレーム内の点灯/消灯パターンを設計し,
RGB各色8bit相当の輝度を表現する手法を概略で説明する.
ビットプレーンごとの重み付けや,実際のフレームレートとの関係を整理し,
どの程度の時間分解能があれば256階調表示が成立するかを述べる.
\begin{figure}[H]
  \centering
  \includegraphics[width=0.8\linewidth]{assets/256gradation.png}
  \caption{RGB各色の点灯回数制御による256階調表現}
  \label{fig:gradation}
\end{figure}

\begin{figure}[H]
  \centering
  \includegraphics[width=0.8\linewidth]{assets/cat_lighting.jpg}
  \caption{RGB各色の点灯回数制御による256階調表現}
  \label{fig:cat_lighting}
\end{figure}

\subsection{32×64画像へのリサイズとマッピング処理}

\begin{figure}[H]
  \centering
  \includegraphics[width=0.8\linewidth]{assets/origin_cat.jpg}
  \caption{元画像}
  \label{fig:origin_cat}
\end{figure}

\begin{figure}[H]
  \centering
  \includegraphics[width=0.8\linewidth]{assets/output_0.45.png}
  \caption{リサイズ後画像}
  \label{fig:resized_cat}
\end{figure}

\subsection{Webサイトからの画像アップロード〜パネル表示までの処理フロー}

本小節では,ユーザがWebブラウザ上のフォームから画像をアップロードしてから,
LEDパネルに表示されるまでの処理フローを時系列に沿って記述する.
HTTPリクエストの受付,ファイル保存,画像変換・リサイズ,フレームバッファ生成,
そしてHUB75パネルへの出力までの各段階でどのような処理を行うかを概要レベルで説明する.
(SONOBE\_poster\_20250718の内容を整理し直して文章化する想定である.)


%======================================================================
% 2.3 表示品質評価
%======================================================================
\section{表示品質評価}

本節では,構築した256階調表示システムを用いて,実際にさまざまな画像を表示した際の
画質について評価する.
再現しやすかった画像と難しかった画像の例を挙げ,人間の目で見た主観的な画質を中心に議論する.

\subsection{はじめに}

\subsection{淡い色が表現しにくい要因の考察}


\subsection{ガンマ補正}

\begin{figure}[H]
  \centering
  \includegraphics[width=0.8\linewidth]{assets/gamma.png}
  \caption{ガンマ補正の原理}
  \label{fig:gamma}
\end{figure}

\begin{figure}[H]
  \centering
  \includegraphics[width=0.8\linewidth]{assets/output_0.45.png}
  \caption{ガンマ0.45補正画像}
  \label{fig:gamma_0.45}
\end{figure}

\begin{figure}[H]
  \centering
  \includegraphics[width=0.8\linewidth]{assets/output_2.2.png}
  \caption{ガンマ2.2補正画像}
  \label{fig:gamma_2.2}
\end{figure}

%======================================================================
% 2.4 小括
%======================================================================
\section{小括}
