%------------------------------------------------------------------
% 付録:LEDマトリクス表示用プログラム
%------------------------------------------------------------------
\appendix
\chapter{LEDマトリクス表示用プログラム}
\label{lbl_appendix_code}

本付録では,第\ref{lbl_chptr3}章で述べたLEDマトリクス表示システムの制御に用いた
C言語プログラムのソースコードを示す.
WiringPiによるGPIO制御,stb\_image系ライブラリによる画像読み込み・リサイズ,
および累積値による256階調表示の実装を含む.

\small
\begin{verbatim}
#define _GNU_SOURCE  // POSIX関数を有効にする
#include <wiringPi.h>
#include <stdio.h>
#include <stdlib.h>
#include <unistd.h>
#include <stdint.h>
#include <stdbool.h>
#include <string.h>
#include <math.h>

// stb_image のインクルード
#define STB_IMAGE_IMPLEMENTATION
#include "stb_image.h"
#define STB_IMAGE_WRITE_IMPLEMENTATION
#include "stb_image_write.h"
#define STB_IMAGE_RESIZE_IMPLEMENTATION
#include "stb_image_resize2.h"

#define NEW_BMP_WIDTH   64
#define NEW_BMP_HEIGHT  32

// スロット周波数(設計値)
#define FREQ_LIGHTING   15360

// HUB75 64x32 は通常 1/16 スキャン(行アドレス0..15)
#define SCAN_ROWS       (NEW_BMP_HEIGHT / 2)

// 通常方式(上段/下段を同時点灯)なので *2 は不要
// 1スロット = 1アドレス行の点灯時間
static inline unsigned int calc_slot_time_us(void) {
    unsigned int t = 1000000u / (FREQ_LIGHTING * SCAN_ROWS);
    if (t == 0) t = 1; // 念のため
    return t;
}

typedef struct { uint8_t r, g, b; } RGB;

// 画像ファイル読み込み(PNG/JPEG/BMP等、stbが自動判別)
unsigned char* load_image(const char* filename, int* out_w, int* out_h) {
    int w, h, ch;
    unsigned char *img = stbi_load(filename, &w, &h, &ch, 3);
    if (!img) {
        fprintf(stderr, "画像読み込み失敗: %s\n", stbi_failure_reason());
        return NULL;
    }
    *out_w = w;
    *out_h = h;
    return img;
}

// リサイズ(RGB構造体配列で返す)
// ※ pow(x, 2.2) による単純ガンマデコード近似を使用
RGB* resize_image(unsigned char* src_rgb, int src_w, int src_h) {
    unsigned char *resized = (unsigned char*)malloc(NEW_BMP_WIDTH * NEW_BMP_HEIGHT * 3);
    if (!resized) return NULL;

    // sRGBを前提にしたリサイズ(ガンマ考慮)
    if (!stbir_resize_uint8_srgb(
            src_rgb, src_w, src_h, 0,
            resized, NEW_BMP_WIDTH, NEW_BMP_HEIGHT, 0, 3)) {
        fprintf(stderr, "リサイズに失敗\n");
        free(resized);
        return NULL;
    }

    RGB *out = (RGB*)malloc(NEW_BMP_WIDTH * NEW_BMP_HEIGHT * sizeof(RGB));
    if (!out) { free(resized); return NULL; }

    for (int i = 0; i < NEW_BMP_WIDTH * NEW_BMP_HEIGHT; i++) {
        double sr = resized[i*3 + 0] / 255.0;
        double sg = resized[i*3 + 1] / 255.0;
        double sb = resized[i*3 + 2] / 255.0;

        int r = (int)lround(pow(sr, 2.2) * 255.0);
        int g = (int)lround(pow(sg, 2.2) * 255.0);
        int b = (int)lround(pow(sb, 2.2) * 255.0);

        if (r < 0) r = 0; if (r > 255) r = 255;
        if (g < 0) g = 0; if (g > 255) g = 255;
        if (b < 0) b = 0; if (b > 255) b = 255;

        out[i].r = (uint8_t)r;
        out[i].g = (uint8_t)g;
        out[i].b = (uint8_t)b;
    }

    free(resized);
    return out;
}

// 点灯判定(累積ディザ / 時間方向誤差拡散)
// 256スロット中に v 回だけONになるよう、スロットkでONかを返す。
// 判定は「床関数差分」: floor(v*(k+1)/256) != floor(v*k/256)
static inline bool accum_dither_on_u8(uint8_t v, uint8_t k) {
    uint16_t a = (uint16_t)v * (uint16_t)(k + 1);
    uint16_t b = (uint16_t)v * (uint16_t)k;
    return (a >> 8) != (b >> 8);
}

// 通常のHUB75制御(上段/下段 同時点灯)
void control_led_matrix(RGB* pixels) {
    // ピン定義(BCM GPIO番号)
    const int pinA     = 22;
    const int pinB     = 23;
    const int pinC     = 24;
    const int pinD     = 25;
    const int pinR1    = 11;
    const int pinR2    = 8;
    const int pinG1    = 27;
    const int pinG2    = 9;
    const int pinB1    = 7;
    const int pinB2    = 10;
    const int pinClock = 17;
    const int pinOE    = 18;
    const int pinLAT   = 4;

    int pins[] = {pinA, pinB, pinC, pinD,
                  pinR1, pinR2, pinG1, pinG2, pinB1, pinB2,
                  pinClock, pinOE, pinLAT};

    for (int p = 0; p < (int)(sizeof(pins)/sizeof(pins[0])); p++) {
        pinMode(pins[p], OUTPUT);
        digitalWrite(pins[p], LOW);
    }

    // OEはHIGHで消灯(一般的な極性)
    digitalWrite(pinOE, HIGH);

    printf("LEDマトリクス制御開始...\n");

    const unsigned int slotTimeUs = calc_slot_time_us();
    printf("slotTime = %u us\n", slotTimeUs);

    uint8_t k = 0; // 0..255

    for (long long n = 0; n < 1000000000LL; n++) {

        // ★この1回のnで「スロット k」を表示する:kはフレーム中固定(縞対策)
        for (int address = 0; address < SCAN_ROWS; address++) {

            // 出力無効化(アドレス変更・シフト中は消灯)
            digitalWrite(pinOE, HIGH);

            // 行アドレス設定(0..15)
            digitalWrite(pinA, (address & 0x1) ? HIGH : LOW);
            digitalWrite(pinB, (address & 0x2) ? HIGH : LOW);
            digitalWrite(pinC, (address & 0x4) ? HIGH : LOW);
            digitalWrite(pinD, (address & 0x8) ? HIGH : LOW);

            const int topRow = address;               // 0..15
            const int botRow = address + SCAN_ROWS;   // 16..31

            // 1行分のシフト(64列)
            for (int x = 0; x < NEW_BMP_WIDTH; x++) {

                // 前列の残りを消す
                digitalWrite(pinR1, LOW);
                digitalWrite(pinG1, LOW);
                digitalWrite(pinB1, LOW);
                digitalWrite(pinR2, LOW);
                digitalWrite(pinG2, LOW);
                digitalWrite(pinB2, LOW);

                RGB t = pixels[topRow * NEW_BMP_WIDTH + x];
                RGB b = pixels[botRow * NEW_BMP_WIDTH + x];

                // 上段(R1/G1/B1)
                if (accum_dither_on_u8(t.r, k)) digitalWrite(pinR1, HIGH);
                if (accum_dither_on_u8(t.g, k)) digitalWrite(pinG1, HIGH);
                if (accum_dither_on_u8(t.b, k)) digitalWrite(pinB1, HIGH);

                // 下段(R2/G2/B2)
                if (accum_dither_on_u8(b.r, k)) digitalWrite(pinR2, HIGH);
                if (accum_dither_on_u8(b.g, k)) digitalWrite(pinG2, HIGH);
                if (accum_dither_on_u8(b.b, k)) digitalWrite(pinB2, HIGH);

                // シフトクロック
                digitalWrite(pinClock, HIGH);
                digitalWrite(pinClock, LOW);
            }

            // ラッチ
            digitalWrite(pinLAT, HIGH);
            digitalWrite(pinLAT, LOW);

            // 出力有効化(点灯)
            digitalWrite(pinOE, LOW);

            delayMicroseconds(slotTimeUs);

            // 次の行へ行く前に確実に消灯
            digitalWrite(pinOE, HIGH);
        }

        // ★フレーム(16行)完了後に k を進める
        k++;
    }

    // 念のため消灯
    digitalWrite(pinOE, HIGH);
}

int main(int argc, char *argv[]) {
    if (argc != 2) {
        fprintf(stderr, "使用方法: %s 入力画像\n", argv[0]);
        return 1;
    }

    if (wiringPiSetupGpio() == -1) {
        fprintf(stderr, "WiringPiの初期化に失敗\n");
        return 1;
    }

    int w, h;
    unsigned char *raw = load_image(argv[1], &w, &h);
    if (!raw) return 1;

    RGB *resized = resize_image(raw, w, h);
    free(raw);
    if (!resized) return 1;

    control_led_matrix(resized);

    free(resized);
    return 0;
}
\end{verbatim}
\normalsize
